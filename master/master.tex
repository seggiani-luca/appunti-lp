
\documentclass[a4paper,11pt]{article}
\usepackage[a4paper, margin=8em]{geometry}

% usa i pacchetti per la scrittura in italiano
\usepackage[french,italian]{babel}
\usepackage[T1]{fontenc}
\usepackage[utf8]{inputenc}
\frenchspacing 

% usa i pacchetti per la formattazione matematica
\usepackage{amsmath, amssymb, amsthm, amsfonts}

% usa altri pacchetti
\usepackage{gensymb}
\usepackage{hyperref}
\usepackage{standalone}

% cose fluttuanti
\usepackage{float}

% imposta il titolo
\title{Appunti Ricerca Operativa}
\author{Luca Seggiani}
\date{2024}

% disegni
\usepackage{pgfplots}
\pgfplotsset{width=10cm,compat=1.9}

% imposta lo stile
% usa helvetica
\usepackage[scaled]{helvet}
% usa palatino
\usepackage{palatino}
% usa un font monospazio guardabile
\usepackage{lmodern}

\renewcommand{\rmdefault}{ppl}
\renewcommand{\sfdefault}{phv}
\renewcommand{\ttdefault}{lmtt}

% disponi il titolo
\makeatletter
\renewcommand{\maketitle} {
	\begin{center} 
		\begin{minipage}[t]{.8\textwidth}
			\textsf{\huge\bfseries \@title} 
		\end{minipage}%
		\begin{minipage}[t]{.2\textwidth}
			\raggedleft \vspace{-1.65em}
			\textsf{\small \@author} \vfill
			\textsf{\small \@date}
		\end{minipage}
		\par
	\end{center}

	\thispagestyle{empty}
	\pagestyle{fancy}
}
\makeatother

% disponi teoremi
\usepackage{tcolorbox}
\newtcolorbox[auto counter, number within=section]{theorem}[2][]{%
	colback=blue!10, 
	colframe=blue!40!black, 
	sharp corners=northwest,
	fonttitle=\sffamily\bfseries, 
	title=Teorema~\thetcbcounter: #2, 
	#1
}

% disponi definizioni
\newtcolorbox[auto counter, number within=section]{definition}[2][]{%
	colback=red!10,
	colframe=red!40!black,
	sharp corners=northwest,
	fonttitle=\sffamily\bfseries,
	title=Definizione~\thetcbcounter: #2,
	#1
}

% disponi problemi
\newtcolorbox[auto counter, number within=section]{problem}[2][]{%
	colback=green!10,
	colframe=green!40!black,
	sharp corners=northwest,
	fonttitle=\sffamily\bfseries,
	title=Problema~\thetcbcounter: #2,
	#1
}

% disponi codice
\usepackage{listings}
\usepackage[table]{xcolor}

\lstdefinestyle{codestyle}{
		backgroundcolor=\color{black!5}, 
		commentstyle=\color{codegreen},
		keywordstyle=\bfseries\color{magenta},
		numberstyle=\sffamily\tiny\color{black!60},
		stringstyle=\color{green!50!black},
		basicstyle=\ttfamily\footnotesize,
		breakatwhitespace=false,         
		breaklines=true,                 
		captionpos=b,                    
		keepspaces=true,                 
		numbers=left,                    
		numbersep=5pt,                  
		showspaces=false,                
		showstringspaces=false,
		showtabs=false,                  
		tabsize=2
}

\lstdefinestyle{shellstyle}{
		backgroundcolor=\color{black!5}, 
		basicstyle=\ttfamily\footnotesize\color{black}, 
		commentstyle=\color{black}, 
		keywordstyle=\color{black},
		numberstyle=\color{black!5},
		stringstyle=\color{black}, 
		showspaces=false,
		showstringspaces=false, 
		showtabs=false, 
		tabsize=2, 
		numbers=none, 
		breaklines=true
}

\lstdefinelanguage{javascript}{
	keywords={typeof, new, true, false, catch, function, return, null, catch, switch, var, if, in, while, do, else, case, break},
	keywordstyle=\color{blue}\bfseries,
	ndkeywords={class, export, boolean, throw, implements, import, this},
	ndkeywordstyle=\color{darkgray}\bfseries,
	identifierstyle=\color{black},
	sensitive=false,
	comment=[l]{//},
	morecomment=[s]{/*}{*/},
	commentstyle=\color{purple}\ttfamily,
	stringstyle=\color{red}\ttfamily,
	morestring=[b]',
	morestring=[b]"
}

% disponi sezioni
\usepackage{titlesec}

\titleformat{\section}
	{\sffamily\Large\bfseries} 
	{\thesection}{1em}{} 
\titleformat{\subsection}
	{\sffamily\large\bfseries}   
	{\thesubsection}{1em}{} 
\titleformat{\subsubsection}
	{\sffamily\normalsize\bfseries} 
	{\thesubsubsection}{1em}{}

% disponi alberi
\usepackage{forest}

\forestset{
	rectstyle/.style={
		for tree={rectangle,draw,font=\large\sffamily}
	},
	roundstyle/.style={
		for tree={circle,draw,font=\large}
	}
}

% disponi algoritmi
\usepackage{algorithm}
\usepackage{algorithmic}
\makeatletter
\renewcommand{\ALG@name}{Algoritmo}
\makeatother

% disponi numeri di pagina
\usepackage{fancyhdr}
\fancyhf{} 
\fancyfoot[L]{\sffamily{\thepage}}

\makeatletter
\fancyhead[L]{\raisebox{1ex}[0pt][0pt]{\sffamily{\@title \ \@date}}} 
\fancyhead[R]{\raisebox{1ex}[0pt][0pt]{\sffamily{\@author}}}
\makeatother

\begin{document}

\pagestyle{fancy}
\thispagestyle{empty}
\renewcommand{\thispagestyle}[1]{}

\maketitle
\documentclass[a4paper,11pt]{article}
\usepackage[a4paper, margin=8em]{geometry}

% usa i pacchetti per la scrittura in italiano
\usepackage[french,italian]{babel}
\usepackage[T1]{fontenc}
\usepackage[utf8]{inputenc}
\frenchspacing 

% usa i pacchetti per la formattazione matematica
\usepackage{amsmath, amssymb, amsthm, amsfonts}

% usa altri pacchetti
\usepackage{gensymb}
\usepackage{hyperref}
\usepackage{standalone}

% imposta il titolo
\title{Appunti Ricerca Operativa}
\author{Luca Seggiani}
\date{23-09-24}

% imposta lo stile
% usa helvetica
\usepackage[scaled]{helvet}
% usa palatino
\usepackage{palatino}
% usa un font monospazio guardabile
\usepackage{lmodern}

\renewcommand{\rmdefault}{ppl}
\renewcommand{\sfdefault}{phv}
\renewcommand{\ttdefault}{lmtt}

% disponi teoremi
\usepackage{tcolorbox}
\newtcolorbox[auto counter, number within=section]{theorem}[2][]{%
	colback=blue!10, 
	colframe=blue!40!black, 
	sharp corners=northwest,
	fonttitle=\sffamily\bfseries, 
	title=Teorema~\thetcbcounter: #2, 
	#1
}

% disponi definizioni
\newtcolorbox[auto counter, number within=section]{definition}[2][]{%
	colback=red!10,
	colframe=red!40!black,
	sharp corners=northwest,
	fonttitle=\sffamily\bfseries,
	title=Definizione~\thetcbcounter: #2,
	#1
}

% disponi problemi
\newtcolorbox[auto counter, number within=section]{problem}[2][]{%
	colback=green!10,
	colframe=green!40!black,
	sharp corners=northwest,
	fonttitle=\sffamily\bfseries,
	title=Problema~\thetcbcounter: #2,
	#1
}

% disponi codice
\usepackage{listings}
\usepackage[table]{xcolor}

\lstdefinestyle{codestyle}{
		backgroundcolor=\color{black!5}, 
		commentstyle=\color{codegreen},
		keywordstyle=\bfseries\color{magenta},
		numberstyle=\sffamily\tiny\color{black!60},
		stringstyle=\color{green!50!black},
		basicstyle=\ttfamily\footnotesize,
		breakatwhitespace=false,         
		breaklines=true,                 
		captionpos=b,                    
		keepspaces=true,                 
		numbers=left,                    
		numbersep=5pt,                  
		showspaces=false,                
		showstringspaces=false,
		showtabs=false,                  
		tabsize=2
}

\lstdefinestyle{shellstyle}{
		backgroundcolor=\color{black!5}, 
		basicstyle=\ttfamily\footnotesize\color{black}, 
		commentstyle=\color{black}, 
		keywordstyle=\color{black},
		numberstyle=\color{black!5},
		stringstyle=\color{black}, 
		showspaces=false,
		showstringspaces=false, 
		showtabs=false, 
		tabsize=2, 
		numbers=none, 
		breaklines=true
}

\lstdefinelanguage{javascript}{
	keywords={typeof, new, true, false, catch, function, return, null, catch, switch, var, if, in, while, do, else, case, break},
	keywordstyle=\color{blue}\bfseries,
	ndkeywords={class, export, boolean, throw, implements, import, this},
	ndkeywordstyle=\color{darkgray}\bfseries,
	identifierstyle=\color{black},
	sensitive=false,
	comment=[l]{//},
	morecomment=[s]{/*}{*/},
	commentstyle=\color{purple}\ttfamily,
	stringstyle=\color{red}\ttfamily,
	morestring=[b]',
	morestring=[b]"
}

% disponi sezioni
\usepackage{titlesec}

\titleformat{\section}
	{\sffamily\Large\bfseries} 
	{\thesection}{1em}{} 
\titleformat{\subsection}
	{\sffamily\large\bfseries}   
	{\thesubsection}{1em}{} 
\titleformat{\subsubsection}
	{\sffamily\normalsize\bfseries} 
	{\thesubsubsection}{1em}{}

% disponi alberi
\usepackage{forest}

\forestset{
	rectstyle/.style={
		for tree={rectangle,draw,font=\large\sffamily}
	},
	roundstyle/.style={
		for tree={circle,draw,font=\large}
	}
}

% disponi algoritmi
\usepackage{algorithm}
\usepackage{algorithmic}
\makeatletter
\renewcommand{\ALG@name}{Algoritmo}
\makeatother

% disponi numeri di pagina
\usepackage{fancyhdr}
\fancyhf{} 
\fancyfoot[L]{\sffamily{\thepage}}

\makeatletter
\fancyhead[L]{\raisebox{1ex}[0pt][0pt]{\sffamily{\@title \ \@date}}} 
\fancyhead[R]{\raisebox{1ex}[0pt][0pt]{\sffamily{\@author}}}
\makeatother

% disegni
\usepackage{pgfplots}
\pgfplotsset{width=10cm,compat=1.9}

\begin{document}
% sezione (data)
\section{Lezione del 23-09-24}

% stili pagina
\thispagestyle{empty}
\pagestyle{fancy}

% testo
\subsection{Introduzione}

\subsubsection{Programma del corso}
Il corso di ricerca operativa si divide in 4 parti:

\begin{enumerate}
	\item Modello di Programmazione Lineare;
	\item Programmazione Lineare su reti, ergo programmazione lineare su grafi;
	\item Programmazione Lineare intera, ergo programmazione lineare col vincolo $x \in \mathbb{Z}^n$;
	\item Programmazione Non Lineare.
\end{enumerate}

Le prime 3 parti hanno come prerequisiti l'algebra lineare: in particolare operazioni matriciali, prodotti scalari, sistemi lineari, teorema di Rouché-Capelli.
La quarta parte richiede invece conoscenze di Analisi II.

\subsubsection{Un problema di programmazione lineare}

La ricerca operativa si occupa di risolvere problemi di ottimizzazione con variabili decisionali e risorse limitate.
Poniamo un problema di esempio:

\begin{problem}{Produzione}
Una ditta produce due prodotti: \textbf{laminato A} e \textbf{laminato B}.
Ogni prodotto deve passare attraverso diversi reparti: il reparto \textbf{materie prime}, il reparto \textbf{taglio}, il reparto \textbf{finiture A} e il reparto \textbf{finiture B}.
Il guadagno è rispettivamente di 8.4 e 11.2 (unità di misura irrilevante) per ogni tipo di laminato.

Ora, nel reparto materie prime, il laminato A occupa 30, ore, e lo B 20 ore.
Nel reparto taglio il laminato A occupa 10 ore e lo B 20 ore.
Il laminato A occupa poi 20 ore nel reparto finiture A, mentre il laminato B occupa 30 ore nel reparto finiture B.
I reparti hanno a disposizione, rispettivamente, 120, 80, 62 e 105 ore.
Possiamo porre queste informazioni in forma tabulare:

	\center \rowcolors{2}{green!10}{green!40!black!20}
	\begin{tabular} { | c || c | c | c | }
		\hline
		\bfseries Reparto & \bfseries Capienza & \bfseries Laminato A & \bfseries Laminato B \\
		\hline 
		Materie prime & 120 & 30 & 20 \\
		Taglio & 80 & 10 & 20 \\
		Finiture A & 62 & 20 & / \\
		Finiture B & 105 & / & 30 \\
		\hline
		\textbf{Guadagno} & & 8.4 & 11.2 \\
		\hline
	\end{tabular}

	\par\bigskip

Quello che ci interessa è chiaramente massimizzare il guadagno.
\end{problem}

Decidiamo di modellizzare questa situazione con un modello matematico.

Il guadagno che abbiamo dai laminati rappresenta una \textbf{funzione obiettivo}, ovvero la funzione che vogliamo ottimizzare.
Ottimizzare significa trovare il modo migliore di massimizzare o minimizzare i valori della funzione agendo sulle variabili decisionali.
La funzione obiettivo va ottimizzata rispettando determinati \textbf{vincoli}, che modellizzano il fatto che le risorse sono limitate.
Una \textbf{soluzione ammissibile} è una qualsiasi soluzione che rispetta i vincoli del problema.
Chiamiamo quindi \textbf{regione ammissibile} l'insieme di tutte le soluzioni ammissibili.
All'interno della regione ammissibile c'è la soluzione che cerchiamo, ovvero la \textbf{soluzione ottima}.

Decidiamo quindi le \textbf{variabili decisionali}, ed esplicitiamo la funzione obiettivo e i vincoli.

In questo caso le variabili decisionali saranno le quantità di laminato A e B da produrre, che individuano un punto in $ \mathbb{R}^2 $ denominato $ ( x_A, x_B ) $. 
Decidere di usare la soluzione $ (1,1) $ significa decidere di produrre 1 unità di laminato A e 1 unità di laminato B, per un guadagno complessivo di $ 8.4 + 11.2 = 19.6 $.

La funzione obiettivo sarà quindi:

$$ f(x_A, x_B) = 8.4 x_A + 11.2 x_B, \quad f: \mathbb{R}^2 \rightarrow \mathbb{R} $$

lineare, e noi saremo interessati a:

$$ \max(f(x_A, x_B)) $$

rispettando i vincoli, ergo nella regione ammissibile.
Per esprimere questi vincoli, cioè il tempo limitato all'interno di ogni reparto, introduciamo il sistema di disequazioni:

\[
	\begin{cases}
		30 x_A + 20 x_B \leq 120 \\
		10 x_A + 20 x_B \leq 80	\\
		20 x_A + 0 x_B \leq 62 \\	
		0 x_A + 30 x_B \leq 105 \\
		- x_A \leq 0 \\
		- x_B \leq 0 \\
	\end{cases}
\]

dove notiamo le ultime due disequazioni indicano la positività di $x_A$ e $x_B$, in forma $ f(x_A, x_B) \leq b $.
Questo sistema non indica altro che la regione ammissibile.

Possiamo riscrivere questo modello usando la notazione dell'algebra lineare.
La funzione obiettiva e i vincoli diventano semplicemente:

\[
	\begin{cases}
		\max(c^T \cdot x) \\
		A \cdot x \leq b	
	\end{cases}
\]

dove $c$ rappresenta il vettore dei costi, $A$ rappresenta la matrice dei costi a $b$ il vettore dei vincoli.
$c$ è trasposto per indicare prodotto fra vettori.

Possiamo scrivere $A$, $b$ e $c$ per esteso:

$$
A:
\begin{pmatrix}
	30 & 20 \\
	10 & 20 \\
	20 & 0 \\
	0 & 30 \\
	-1 & 0 \\
	0 & -1
\end{pmatrix}, \quad
b:
\begin{pmatrix}
	120 \\
	80 \\
	62 \\ 
	105 \\ 
	0 \\ 
	0 
\end{pmatrix}, \quad 
c:
\begin{pmatrix}
	8.4 \\
	11.2 \\
\end{pmatrix}
$$

Notiamo come $A$ e $b$ hanno dimensione verticale $ 4 + 2 = 6 $, dai 4 vincoli superiori e i 2 vincoli inferiori.

A questo punto, possiamo disegnare la regione ammissibile come l'intersezione dei semipiani individuati da ogni singola disuguaglianza.

Si riporta un grafico:

\begin{center}

\begin{tikzpicture}
\begin{axis}[
    axis lines = middle,
    xlabel = {$x_A$},
    ylabel = {$x_B$},
    xmin=0, xmax=7.9,
    ymin=0, ymax=3.9,
    domain=0:10,
    samples=100,
    width=15cm, height=7cm,
    legend pos=north east
  ]

% regione ammissibile

	\addplot[fill=gray, opacity=0.4, forget plot] 
    coordinates {
			(0, 0)
			(3.1, 0)
			(3.1, 1.35)
			(2,3)
			(1, 3.5)
			(0, 3.5)
		};

% rette

\addplot[domain=2:3.1, thick, blue] {6 - 1.5*x}; 
\addlegendentry{$ 30 x_A + 20 x_B \leq 120 $}

\addplot[domain=1:2, thick, green] {4 - 0.5*x}; 
\addlegendentry{$ 10 x_A + 20 x_B \leq 80 $}

\addplot[thick, purple] coordinates {(3.1, 0) (3.1, 1.35)};
\addlegendentry{$ 20 x_A + 0 x_B \leq 62 $}

\addplot[domain=0:1, thick, red] {3.5}; 
\addlegendentry{$ 0 x_A + 30 x_B \leq 105 $}
	
\end{axis}
\end{tikzpicture}

\end{center}

In diversi colori sono riportati i margini delle disequazioni, mentre in grigio è evidenziata la regione ammissibile.
Qualsiasi punto all'interno della regione ammissibile vale come soluzione, e almeno uno di essi è soluzione ottimale.

\par\smallskip

Il modello finora descritto prende il nome di modello di programmazione lineare, e permette di formulare problemi di programmazione lineare (LP).

\begin{definition}{Problema di programmazione lineare (1)}
Un problema di programmazione lineare (LP) riguarda l'ottimizzazione di una funzione lineare in più variabili
soggetta a vincoli di tipo $ =, \ \leq $ e $ \geq $, ovvero in forma:
\[
	\begin{cases}
			\min / \max(c^T \cdot x) \\
			A_i x \leq b \\
			B_j x \geq d \\
			C_k x = e \\
	\end{cases}
\]
\end{definition}

"Programmazione" qui non ha alcun legame col concetto di programmazione informatica, ma si riferisce al fatto che il modello è effettivamente \textit{programmabile}.

"Lineare" si riferisce alla linearità dei problemi che ci permette di risolvere (e quindi del modello).




\end{document}

\documentclass[a4paper,11pt]{article}
\usepackage[a4paper, margin=8em]{geometry}

% usa i pacchetti per la scrittura in italiano
\usepackage[french,italian]{babel}
\usepackage[T1]{fontenc}
\usepackage[utf8]{inputenc}
\frenchspacing 

% usa i pacchetti per la formattazione matematica
\usepackage{amsmath, amssymb, amsthm, amsfonts}

% usa altri pacchetti
\usepackage{gensymb}
\usepackage{hyperref}
\usepackage{standalone}

% imposta il titolo
\title{Appunti Ricerca Operativa}
\author{Luca Seggiani}
\date{23-09-24}

% imposta lo stile
% usa helvetica
\usepackage[scaled]{helvet}
% usa palatino
\usepackage{palatino}
% usa un font monospazio guardabile
\usepackage{lmodern}

\renewcommand{\rmdefault}{ppl}
\renewcommand{\sfdefault}{phv}
\renewcommand{\ttdefault}{lmtt}

% disponi teoremi
\usepackage{tcolorbox}
\newtcolorbox[auto counter, number within=section]{theorem}[2][]{%
	colback=blue!10, 
	colframe=blue!40!black, 
	sharp corners=northwest,
	fonttitle=\sffamily\bfseries, 
	title=Teorema~\thetcbcounter: #2, 
	#1
}

% disponi definizioni
\newtcolorbox[auto counter, number within=section]{definition}[2][]{%
	colback=red!10,
	colframe=red!40!black,
	sharp corners=northwest,
	fonttitle=\sffamily\bfseries,
	title=Definizione~\thetcbcounter: #2,
	#1
}

% disponi problemi
\newtcolorbox[auto counter, number within=section]{problem}[2][]{%
	colback=green!10,
	colframe=green!40!black,
	sharp corners=northwest,
	fonttitle=\sffamily\bfseries,
	title=Problema~\thetcbcounter: #2,
	#1
}

% disponi codice
\usepackage{listings}
\usepackage[table]{xcolor}

\lstdefinestyle{codestyle}{
		backgroundcolor=\color{black!5}, 
		commentstyle=\color{codegreen},
		keywordstyle=\bfseries\color{magenta},
		numberstyle=\sffamily\tiny\color{black!60},
		stringstyle=\color{green!50!black},
		basicstyle=\ttfamily\footnotesize,
		breakatwhitespace=false,         
		breaklines=true,                 
		captionpos=b,                    
		keepspaces=true,                 
		numbers=left,                    
		numbersep=5pt,                  
		showspaces=false,                
		showstringspaces=false,
		showtabs=false,                  
		tabsize=2
}

\lstdefinestyle{shellstyle}{
		backgroundcolor=\color{black!5}, 
		basicstyle=\ttfamily\footnotesize\color{black}, 
		commentstyle=\color{black}, 
		keywordstyle=\color{black},
		numberstyle=\color{black!5},
		stringstyle=\color{black}, 
		showspaces=false,
		showstringspaces=false, 
		showtabs=false, 
		tabsize=2, 
		numbers=none, 
		breaklines=true
}

\lstdefinelanguage{javascript}{
	keywords={typeof, new, true, false, catch, function, return, null, catch, switch, var, if, in, while, do, else, case, break},
	keywordstyle=\color{blue}\bfseries,
	ndkeywords={class, export, boolean, throw, implements, import, this},
	ndkeywordstyle=\color{darkgray}\bfseries,
	identifierstyle=\color{black},
	sensitive=false,
	comment=[l]{//},
	morecomment=[s]{/*}{*/},
	commentstyle=\color{purple}\ttfamily,
	stringstyle=\color{red}\ttfamily,
	morestring=[b]',
	morestring=[b]"
}

% disponi sezioni
\usepackage{titlesec}

\titleformat{\section}
	{\sffamily\Large\bfseries} 
	{\thesection}{1em}{} 
\titleformat{\subsection}
	{\sffamily\large\bfseries}   
	{\thesubsection}{1em}{} 
\titleformat{\subsubsection}
	{\sffamily\normalsize\bfseries} 
	{\thesubsubsection}{1em}{}

% disponi alberi
\usepackage{forest}

\forestset{
	rectstyle/.style={
		for tree={rectangle,draw,font=\large\sffamily}
	},
	roundstyle/.style={
		for tree={circle,draw,font=\large}
	}
}

% disponi algoritmi
\usepackage{algorithm}
\usepackage{algorithmic}
\makeatletter
\renewcommand{\ALG@name}{Algoritmo}
\makeatother

% disponi numeri di pagina
\usepackage{fancyhdr}
\fancyhf{} 
\fancyfoot[L]{\sffamily{\thepage}}

\makeatletter
\fancyhead[L]{\raisebox{1ex}[0pt][0pt]{\sffamily{\@title \ \@date}}} 
\fancyhead[R]{\raisebox{1ex}[0pt][0pt]{\sffamily{\@author}}}
\makeatother

% disegni
\usepackage{pgfplots}
\pgfplotsset{width=10cm,compat=1.9}

\begin{document}
% sezione (data)
\section{Lezione del 23-09-24}

% stili pagina
\thispagestyle{empty}
\pagestyle{fancy}

% testo
\subsection{Introduzione}

\subsubsection{Programma del corso}
Il corso di ricerca operativa si divide in 4 parti:

\begin{enumerate}
	\item Modello di Programmazione Lineare;
	\item Programmazione Lineare su reti, ergo programmazione lineare su grafi;
	\item Programmazione Lineare intera, ergo programmazione lineare col vincolo $x \in \mathbb{Z}^n$;
	\item Programmazione Non Lineare.
\end{enumerate}

Le prime 3 parti hanno come prerequisiti l'algebra lineare: in particolare operazioni matriciali, prodotti scalari, sistemi lineari, teorema di Rouché-Capelli.
La quarta parte richiede invece conoscenze di Analisi II.

\subsubsection{Un problema di programmazione lineare}

La ricerca operativa si occupa di risolvere problemi di ottimizzazione con variabili decisionali e risorse limitate.
Poniamo un problema di esempio:

\begin{problem}{Produzione}
Una ditta produce due prodotti: \textbf{laminato A} e \textbf{laminato B}.
Ogni prodotto deve passare attraverso diversi reparti: il reparto \textbf{materie prime}, il reparto \textbf{taglio}, il reparto \textbf{finiture A} e il reparto \textbf{finiture B}.
Il guadagno è rispettivamente di 8.4 e 11.2 (unità di misura irrilevante) per ogni tipo di laminato.

Ora, nel reparto materie prime, il laminato A occupa 30, ore, e lo B 20 ore.
Nel reparto taglio il laminato A occupa 10 ore e lo B 20 ore.
Il laminato A occupa poi 20 ore nel reparto finiture A, mentre il laminato B occupa 30 ore nel reparto finiture B.
I reparti hanno a disposizione, rispettivamente, 120, 80, 62 e 105 ore.
Possiamo porre queste informazioni in forma tabulare:

	\center \rowcolors{2}{green!10}{green!40!black!20}
	\begin{tabular} { | c || c | c | c | }
		\hline
		\bfseries Reparto & \bfseries Capienza & \bfseries Laminato A & \bfseries Laminato B \\
		\hline 
		Materie prime & 120 & 30 & 20 \\
		Taglio & 80 & 10 & 20 \\
		Finiture A & 62 & 20 & / \\
		Finiture B & 105 & / & 30 \\
		\hline
		\textbf{Guadagno} & & 8.4 & 11.2 \\
		\hline
	\end{tabular}

	\par\bigskip

Quello che ci interessa è chiaramente massimizzare il guadagno.
\end{problem}

Decidiamo di modellizzare questa situazione con un modello matematico.

Il guadagno che abbiamo dai laminati rappresenta una \textbf{funzione obiettivo}, ovvero la funzione che vogliamo ottimizzare.
Ottimizzare significa trovare il modo migliore di massimizzare o minimizzare i valori della funzione agendo sulle variabili decisionali.
La funzione obiettivo va ottimizzata rispettando determinati \textbf{vincoli}, che modellizzano il fatto che le risorse sono limitate.
Una \textbf{soluzione ammissibile} è una qualsiasi soluzione che rispetta i vincoli del problema.
Chiamiamo quindi \textbf{regione ammissibile} l'insieme di tutte le soluzioni ammissibili.
All'interno della regione ammissibile c'è la soluzione che cerchiamo, ovvero la \textbf{soluzione ottima}.

Decidiamo quindi le \textbf{variabili decisionali}, ed esplicitiamo la funzione obiettivo e i vincoli.

In questo caso le variabili decisionali saranno le quantità di laminato A e B da produrre, che individuano un punto in $ \mathbb{R}^2 $ denominato $ ( x_A, x_B ) $. 
Decidere di usare la soluzione $ (1,1) $ significa decidere di produrre 1 unità di laminato A e 1 unità di laminato B, per un guadagno complessivo di $ 8.4 + 11.2 = 19.6 $.

La funzione obiettivo sarà quindi:

$$ f(x_A, x_B) = 8.4 x_A + 11.2 x_B, \quad f: \mathbb{R}^2 \rightarrow \mathbb{R} $$

lineare, e noi saremo interessati a:

$$ \max(f(x_A, x_B)) $$

rispettando i vincoli, ergo nella regione ammissibile.
Per esprimere questi vincoli, cioè il tempo limitato all'interno di ogni reparto, introduciamo il sistema di disequazioni:

\[
	\begin{cases}
		30 x_A + 20 x_B \leq 120 \\
		10 x_A + 20 x_B \leq 80	\\
		20 x_A + 0 x_B \leq 62 \\	
		0 x_A + 30 x_B \leq 105 \\
		- x_A \leq 0 \\
		- x_B \leq 0 \\
	\end{cases}
\]

dove notiamo le ultime due disequazioni indicano la positività di $x_A$ e $x_B$, in forma $ f(x_A, x_B) \leq b $.
Questo sistema non indica altro che la regione ammissibile.

Possiamo riscrivere questo modello usando la notazione dell'algebra lineare.
La funzione obiettiva e i vincoli diventano semplicemente:

\[
	\begin{cases}
		\max(c^T \cdot x) \\
		A \cdot x \leq b	
	\end{cases}
\]

dove $c$ rappresenta il vettore dei costi, $A$ rappresenta la matrice dei costi a $b$ il vettore dei vincoli.
$c$ è trasposto per indicare prodotto fra vettori.

Possiamo scrivere $A$, $b$ e $c$ per esteso:

$$
A:
\begin{pmatrix}
	30 & 20 \\
	10 & 20 \\
	20 & 0 \\
	0 & 30 \\
	-1 & 0 \\
	0 & -1
\end{pmatrix}, \quad
b:
\begin{pmatrix}
	120 \\
	80 \\
	62 \\ 
	105 \\ 
	0 \\ 
	0 
\end{pmatrix}, \quad 
c:
\begin{pmatrix}
	8.4 \\
	11.2 \\
\end{pmatrix}
$$

Notiamo come $A$ e $b$ hanno dimensione verticale $ 4 + 2 = 6 $, dai 4 vincoli superiori e i 2 vincoli inferiori.

A questo punto, possiamo disegnare la regione ammissibile come l'intersezione dei semipiani individuati da ogni singola disuguaglianza.

Si riporta un grafico:

\begin{center}

\begin{tikzpicture}
\begin{axis}[
    axis lines = middle,
    xlabel = {$x_A$},
    ylabel = {$x_B$},
    xmin=0, xmax=7.9,
    ymin=0, ymax=3.9,
    domain=0:10,
    samples=100,
    width=15cm, height=7cm,
    legend pos=north east
  ]

% regione ammissibile

	\addplot[fill=gray, opacity=0.4, forget plot] 
    coordinates {
			(0, 0)
			(3.1, 0)
			(3.1, 1.35)
			(2,3)
			(1, 3.5)
			(0, 3.5)
		};

% rette

\addplot[domain=2:3.1, thick, blue] {6 - 1.5*x}; 
\addlegendentry{$ 30 x_A + 20 x_B \leq 120 $}

\addplot[domain=1:2, thick, green] {4 - 0.5*x}; 
\addlegendentry{$ 10 x_A + 20 x_B \leq 80 $}

\addplot[thick, purple] coordinates {(3.1, 0) (3.1, 1.35)};
\addlegendentry{$ 20 x_A + 0 x_B \leq 62 $}

\addplot[domain=0:1, thick, red] {3.5}; 
\addlegendentry{$ 0 x_A + 30 x_B \leq 105 $}
	
\end{axis}
\end{tikzpicture}

\end{center}

In diversi colori sono riportati i margini delle disequazioni, mentre in grigio è evidenziata la regione ammissibile.
Qualsiasi punto all'interno della regione ammissibile vale come soluzione, e almeno uno di essi è soluzione ottimale.

\par\smallskip

Il modello finora descritto prende il nome di modello di programmazione lineare, e permette di formulare problemi di programmazione lineare (LP).

\begin{definition}{Problema di programmazione lineare (1)}
Un problema di programmazione lineare (LP) riguarda l'ottimizzazione di una funzione lineare in più variabili
soggetta a vincoli di tipo $ =, \ \leq $ e $ \geq $, ovvero in forma:
\[
	\begin{cases}
			\min / \max(c^T \cdot x) \\
			A_i x \leq b \\
			B_j x \geq d \\
			C_k x = e \\
	\end{cases}
\]
\end{definition}

"Programmazione" qui non ha alcun legame col concetto di programmazione informatica, ma si riferisce al fatto che il modello è effettivamente \textit{programmabile}.

"Lineare" si riferisce alla linearità dei problemi che ci permette di risolvere (e quindi del modello).




\end{document}

\documentclass[a4paper,11pt]{article}
\usepackage[a4paper, margin=8em]{geometry}

% usa i pacchetti per la scrittura in italiano
\usepackage[french,italian]{babel}
\usepackage[T1]{fontenc}
\usepackage[utf8]{inputenc}
\frenchspacing 

% usa i pacchetti per la formattazione matematica
\usepackage{amsmath, amssymb, amsthm, amsfonts}

% usa altri pacchetti
\usepackage{gensymb}
\usepackage{hyperref}
\usepackage{standalone}

% imposta il titolo
\title{Appunti Ricerca Operativa}
\author{Luca Seggiani}
\date{23-09-24}

% imposta lo stile
% usa helvetica
\usepackage[scaled]{helvet}
% usa palatino
\usepackage{palatino}
% usa un font monospazio guardabile
\usepackage{lmodern}

\renewcommand{\rmdefault}{ppl}
\renewcommand{\sfdefault}{phv}
\renewcommand{\ttdefault}{lmtt}

% disponi teoremi
\usepackage{tcolorbox}
\newtcolorbox[auto counter, number within=section]{theorem}[2][]{%
	colback=blue!10, 
	colframe=blue!40!black, 
	sharp corners=northwest,
	fonttitle=\sffamily\bfseries, 
	title=Teorema~\thetcbcounter: #2, 
	#1
}

% disponi definizioni
\newtcolorbox[auto counter, number within=section]{definition}[2][]{%
	colback=red!10,
	colframe=red!40!black,
	sharp corners=northwest,
	fonttitle=\sffamily\bfseries,
	title=Definizione~\thetcbcounter: #2,
	#1
}

% disponi problemi
\newtcolorbox[auto counter, number within=section]{problem}[2][]{%
	colback=green!10,
	colframe=green!40!black,
	sharp corners=northwest,
	fonttitle=\sffamily\bfseries,
	title=Problema~\thetcbcounter: #2,
	#1
}

% disponi codice
\usepackage{listings}
\usepackage[table]{xcolor}

\lstdefinestyle{codestyle}{
		backgroundcolor=\color{black!5}, 
		commentstyle=\color{codegreen},
		keywordstyle=\bfseries\color{magenta},
		numberstyle=\sffamily\tiny\color{black!60},
		stringstyle=\color{green!50!black},
		basicstyle=\ttfamily\footnotesize,
		breakatwhitespace=false,         
		breaklines=true,                 
		captionpos=b,                    
		keepspaces=true,                 
		numbers=left,                    
		numbersep=5pt,                  
		showspaces=false,                
		showstringspaces=false,
		showtabs=false,                  
		tabsize=2
}

\lstdefinestyle{shellstyle}{
		backgroundcolor=\color{black!5}, 
		basicstyle=\ttfamily\footnotesize\color{black}, 
		commentstyle=\color{black}, 
		keywordstyle=\color{black},
		numberstyle=\color{black!5},
		stringstyle=\color{black}, 
		showspaces=false,
		showstringspaces=false, 
		showtabs=false, 
		tabsize=2, 
		numbers=none, 
		breaklines=true
}

\lstdefinelanguage{javascript}{
	keywords={typeof, new, true, false, catch, function, return, null, catch, switch, var, if, in, while, do, else, case, break},
	keywordstyle=\color{blue}\bfseries,
	ndkeywords={class, export, boolean, throw, implements, import, this},
	ndkeywordstyle=\color{darkgray}\bfseries,
	identifierstyle=\color{black},
	sensitive=false,
	comment=[l]{//},
	morecomment=[s]{/*}{*/},
	commentstyle=\color{purple}\ttfamily,
	stringstyle=\color{red}\ttfamily,
	morestring=[b]',
	morestring=[b]"
}

% disponi sezioni
\usepackage{titlesec}

\titleformat{\section}
	{\sffamily\Large\bfseries} 
	{\thesection}{1em}{} 
\titleformat{\subsection}
	{\sffamily\large\bfseries}   
	{\thesubsection}{1em}{} 
\titleformat{\subsubsection}
	{\sffamily\normalsize\bfseries} 
	{\thesubsubsection}{1em}{}

% disponi alberi
\usepackage{forest}

\forestset{
	rectstyle/.style={
		for tree={rectangle,draw,font=\large\sffamily}
	},
	roundstyle/.style={
		for tree={circle,draw,font=\large}
	}
}

% disponi algoritmi
\usepackage{algorithm}
\usepackage{algorithmic}
\makeatletter
\renewcommand{\ALG@name}{Algoritmo}
\makeatother

% disponi numeri di pagina
\usepackage{fancyhdr}
\fancyhf{} 
\fancyfoot[L]{\sffamily{\thepage}}

\makeatletter
\fancyhead[L]{\raisebox{1ex}[0pt][0pt]{\sffamily{\@title \ \@date}}} 
\fancyhead[R]{\raisebox{1ex}[0pt][0pt]{\sffamily{\@author}}}
\makeatother

% disegni
\usepackage{pgfplots}
\pgfplotsset{width=10cm,compat=1.9}

\begin{document}
% sezione (data)
\section{Lezione del 23-09-24}

% stili pagina
\thispagestyle{empty}
\pagestyle{fancy}

% testo
\subsection{Introduzione}

\subsubsection{Programma del corso}
Il corso di ricerca operativa si divide in 4 parti:

\begin{enumerate}
	\item Modello di Programmazione Lineare;
	\item Programmazione Lineare su reti, ergo programmazione lineare su grafi;
	\item Programmazione Lineare intera, ergo programmazione lineare col vincolo $x \in \mathbb{Z}^n$;
	\item Programmazione Non Lineare.
\end{enumerate}

Le prime 3 parti hanno come prerequisiti l'algebra lineare: in particolare operazioni matriciali, prodotti scalari, sistemi lineari, teorema di Rouché-Capelli.
La quarta parte richiede invece conoscenze di Analisi II.

\subsubsection{Un problema di programmazione lineare}

La ricerca operativa si occupa di risolvere problemi di ottimizzazione con variabili decisionali e risorse limitate.
Poniamo un problema di esempio:

\begin{problem}{Produzione}
Una ditta produce due prodotti: \textbf{laminato A} e \textbf{laminato B}.
Ogni prodotto deve passare attraverso diversi reparti: il reparto \textbf{materie prime}, il reparto \textbf{taglio}, il reparto \textbf{finiture A} e il reparto \textbf{finiture B}.
Il guadagno è rispettivamente di 8.4 e 11.2 (unità di misura irrilevante) per ogni tipo di laminato.

Ora, nel reparto materie prime, il laminato A occupa 30, ore, e lo B 20 ore.
Nel reparto taglio il laminato A occupa 10 ore e lo B 20 ore.
Il laminato A occupa poi 20 ore nel reparto finiture A, mentre il laminato B occupa 30 ore nel reparto finiture B.
I reparti hanno a disposizione, rispettivamente, 120, 80, 62 e 105 ore.
Possiamo porre queste informazioni in forma tabulare:

	\center \rowcolors{2}{green!10}{green!40!black!20}
	\begin{tabular} { | c || c | c | c | }
		\hline
		\bfseries Reparto & \bfseries Capienza & \bfseries Laminato A & \bfseries Laminato B \\
		\hline 
		Materie prime & 120 & 30 & 20 \\
		Taglio & 80 & 10 & 20 \\
		Finiture A & 62 & 20 & / \\
		Finiture B & 105 & / & 30 \\
		\hline
		\textbf{Guadagno} & & 8.4 & 11.2 \\
		\hline
	\end{tabular}

	\par\bigskip

Quello che ci interessa è chiaramente massimizzare il guadagno.
\end{problem}

Decidiamo di modellizzare questa situazione con un modello matematico.

Il guadagno che abbiamo dai laminati rappresenta una \textbf{funzione obiettivo}, ovvero la funzione che vogliamo ottimizzare.
Ottimizzare significa trovare il modo migliore di massimizzare o minimizzare i valori della funzione agendo sulle variabili decisionali.
La funzione obiettivo va ottimizzata rispettando determinati \textbf{vincoli}, che modellizzano il fatto che le risorse sono limitate.
Una \textbf{soluzione ammissibile} è una qualsiasi soluzione che rispetta i vincoli del problema.
Chiamiamo quindi \textbf{regione ammissibile} l'insieme di tutte le soluzioni ammissibili.
All'interno della regione ammissibile c'è la soluzione che cerchiamo, ovvero la \textbf{soluzione ottima}.

Decidiamo quindi le \textbf{variabili decisionali}, ed esplicitiamo la funzione obiettivo e i vincoli.

In questo caso le variabili decisionali saranno le quantità di laminato A e B da produrre, che individuano un punto in $ \mathbb{R}^2 $ denominato $ ( x_A, x_B ) $. 
Decidere di usare la soluzione $ (1,1) $ significa decidere di produrre 1 unità di laminato A e 1 unità di laminato B, per un guadagno complessivo di $ 8.4 + 11.2 = 19.6 $.

La funzione obiettivo sarà quindi:

$$ f(x_A, x_B) = 8.4 x_A + 11.2 x_B, \quad f: \mathbb{R}^2 \rightarrow \mathbb{R} $$

lineare, e noi saremo interessati a:

$$ \max(f(x_A, x_B)) $$

rispettando i vincoli, ergo nella regione ammissibile.
Per esprimere questi vincoli, cioè il tempo limitato all'interno di ogni reparto, introduciamo il sistema di disequazioni:

\[
	\begin{cases}
		30 x_A + 20 x_B \leq 120 \\
		10 x_A + 20 x_B \leq 80	\\
		20 x_A + 0 x_B \leq 62 \\	
		0 x_A + 30 x_B \leq 105 \\
		- x_A \leq 0 \\
		- x_B \leq 0 \\
	\end{cases}
\]

dove notiamo le ultime due disequazioni indicano la positività di $x_A$ e $x_B$, in forma $ f(x_A, x_B) \leq b $.
Questo sistema non indica altro che la regione ammissibile.

Possiamo riscrivere questo modello usando la notazione dell'algebra lineare.
La funzione obiettiva e i vincoli diventano semplicemente:

\[
	\begin{cases}
		\max(c^T \cdot x) \\
		A \cdot x \leq b	
	\end{cases}
\]

dove $c$ rappresenta il vettore dei costi, $A$ rappresenta la matrice dei costi a $b$ il vettore dei vincoli.
$c$ è trasposto per indicare prodotto fra vettori.

Possiamo scrivere $A$, $b$ e $c$ per esteso:

$$
A:
\begin{pmatrix}
	30 & 20 \\
	10 & 20 \\
	20 & 0 \\
	0 & 30 \\
	-1 & 0 \\
	0 & -1
\end{pmatrix}, \quad
b:
\begin{pmatrix}
	120 \\
	80 \\
	62 \\ 
	105 \\ 
	0 \\ 
	0 
\end{pmatrix}, \quad 
c:
\begin{pmatrix}
	8.4 \\
	11.2 \\
\end{pmatrix}
$$

Notiamo come $A$ e $b$ hanno dimensione verticale $ 4 + 2 = 6 $, dai 4 vincoli superiori e i 2 vincoli inferiori.

A questo punto, possiamo disegnare la regione ammissibile come l'intersezione dei semipiani individuati da ogni singola disuguaglianza.

Si riporta un grafico:

\begin{center}

\begin{tikzpicture}
\begin{axis}[
    axis lines = middle,
    xlabel = {$x_A$},
    ylabel = {$x_B$},
    xmin=0, xmax=7.9,
    ymin=0, ymax=3.9,
    domain=0:10,
    samples=100,
    width=15cm, height=7cm,
    legend pos=north east
  ]

% regione ammissibile

	\addplot[fill=gray, opacity=0.4, forget plot] 
    coordinates {
			(0, 0)
			(3.1, 0)
			(3.1, 1.35)
			(2,3)
			(1, 3.5)
			(0, 3.5)
		};

% rette

\addplot[domain=2:3.1, thick, blue] {6 - 1.5*x}; 
\addlegendentry{$ 30 x_A + 20 x_B \leq 120 $}

\addplot[domain=1:2, thick, green] {4 - 0.5*x}; 
\addlegendentry{$ 10 x_A + 20 x_B \leq 80 $}

\addplot[thick, purple] coordinates {(3.1, 0) (3.1, 1.35)};
\addlegendentry{$ 20 x_A + 0 x_B \leq 62 $}

\addplot[domain=0:1, thick, red] {3.5}; 
\addlegendentry{$ 0 x_A + 30 x_B \leq 105 $}
	
\end{axis}
\end{tikzpicture}

\end{center}

In diversi colori sono riportati i margini delle disequazioni, mentre in grigio è evidenziata la regione ammissibile.
Qualsiasi punto all'interno della regione ammissibile vale come soluzione, e almeno uno di essi è soluzione ottimale.

\par\smallskip

Il modello finora descritto prende il nome di modello di programmazione lineare, e permette di formulare problemi di programmazione lineare (LP).

\begin{definition}{Problema di programmazione lineare (1)}
Un problema di programmazione lineare (LP) riguarda l'ottimizzazione di una funzione lineare in più variabili
soggetta a vincoli di tipo $ =, \ \leq $ e $ \geq $, ovvero in forma:
\[
	\begin{cases}
			\min / \max(c^T \cdot x) \\
			A_i x \leq b \\
			B_j x \geq d \\
			C_k x = e \\
	\end{cases}
\]
\end{definition}

"Programmazione" qui non ha alcun legame col concetto di programmazione informatica, ma si riferisce al fatto che il modello è effettivamente \textit{programmabile}.

"Lineare" si riferisce alla linearità dei problemi che ci permette di risolvere (e quindi del modello).




\end{document}

\documentclass[a4paper,11pt]{article}
\usepackage[a4paper, margin=8em]{geometry}

% usa i pacchetti per la scrittura in italiano
\usepackage[french,italian]{babel}
\usepackage[T1]{fontenc}
\usepackage[utf8]{inputenc}
\frenchspacing 

% usa i pacchetti per la formattazione matematica
\usepackage{amsmath, amssymb, amsthm, amsfonts}

% usa altri pacchetti
\usepackage{gensymb}
\usepackage{hyperref}
\usepackage{standalone}

% imposta il titolo
\title{Appunti Ricerca Operativa}
\author{Luca Seggiani}
\date{23-09-24}

% imposta lo stile
% usa helvetica
\usepackage[scaled]{helvet}
% usa palatino
\usepackage{palatino}
% usa un font monospazio guardabile
\usepackage{lmodern}

\renewcommand{\rmdefault}{ppl}
\renewcommand{\sfdefault}{phv}
\renewcommand{\ttdefault}{lmtt}

% disponi teoremi
\usepackage{tcolorbox}
\newtcolorbox[auto counter, number within=section]{theorem}[2][]{%
	colback=blue!10, 
	colframe=blue!40!black, 
	sharp corners=northwest,
	fonttitle=\sffamily\bfseries, 
	title=Teorema~\thetcbcounter: #2, 
	#1
}

% disponi definizioni
\newtcolorbox[auto counter, number within=section]{definition}[2][]{%
	colback=red!10,
	colframe=red!40!black,
	sharp corners=northwest,
	fonttitle=\sffamily\bfseries,
	title=Definizione~\thetcbcounter: #2,
	#1
}

% disponi problemi
\newtcolorbox[auto counter, number within=section]{problem}[2][]{%
	colback=green!10,
	colframe=green!40!black,
	sharp corners=northwest,
	fonttitle=\sffamily\bfseries,
	title=Problema~\thetcbcounter: #2,
	#1
}

% disponi codice
\usepackage{listings}
\usepackage[table]{xcolor}

\lstdefinestyle{codestyle}{
		backgroundcolor=\color{black!5}, 
		commentstyle=\color{codegreen},
		keywordstyle=\bfseries\color{magenta},
		numberstyle=\sffamily\tiny\color{black!60},
		stringstyle=\color{green!50!black},
		basicstyle=\ttfamily\footnotesize,
		breakatwhitespace=false,         
		breaklines=true,                 
		captionpos=b,                    
		keepspaces=true,                 
		numbers=left,                    
		numbersep=5pt,                  
		showspaces=false,                
		showstringspaces=false,
		showtabs=false,                  
		tabsize=2
}

\lstdefinestyle{shellstyle}{
		backgroundcolor=\color{black!5}, 
		basicstyle=\ttfamily\footnotesize\color{black}, 
		commentstyle=\color{black}, 
		keywordstyle=\color{black},
		numberstyle=\color{black!5},
		stringstyle=\color{black}, 
		showspaces=false,
		showstringspaces=false, 
		showtabs=false, 
		tabsize=2, 
		numbers=none, 
		breaklines=true
}

\lstdefinelanguage{javascript}{
	keywords={typeof, new, true, false, catch, function, return, null, catch, switch, var, if, in, while, do, else, case, break},
	keywordstyle=\color{blue}\bfseries,
	ndkeywords={class, export, boolean, throw, implements, import, this},
	ndkeywordstyle=\color{darkgray}\bfseries,
	identifierstyle=\color{black},
	sensitive=false,
	comment=[l]{//},
	morecomment=[s]{/*}{*/},
	commentstyle=\color{purple}\ttfamily,
	stringstyle=\color{red}\ttfamily,
	morestring=[b]',
	morestring=[b]"
}

% disponi sezioni
\usepackage{titlesec}

\titleformat{\section}
	{\sffamily\Large\bfseries} 
	{\thesection}{1em}{} 
\titleformat{\subsection}
	{\sffamily\large\bfseries}   
	{\thesubsection}{1em}{} 
\titleformat{\subsubsection}
	{\sffamily\normalsize\bfseries} 
	{\thesubsubsection}{1em}{}

% disponi alberi
\usepackage{forest}

\forestset{
	rectstyle/.style={
		for tree={rectangle,draw,font=\large\sffamily}
	},
	roundstyle/.style={
		for tree={circle,draw,font=\large}
	}
}

% disponi algoritmi
\usepackage{algorithm}
\usepackage{algorithmic}
\makeatletter
\renewcommand{\ALG@name}{Algoritmo}
\makeatother

% disponi numeri di pagina
\usepackage{fancyhdr}
\fancyhf{} 
\fancyfoot[L]{\sffamily{\thepage}}

\makeatletter
\fancyhead[L]{\raisebox{1ex}[0pt][0pt]{\sffamily{\@title \ \@date}}} 
\fancyhead[R]{\raisebox{1ex}[0pt][0pt]{\sffamily{\@author}}}
\makeatother

% disegni
\usepackage{pgfplots}
\pgfplotsset{width=10cm,compat=1.9}

\begin{document}
% sezione (data)
\section{Lezione del 23-09-24}

% stili pagina
\thispagestyle{empty}
\pagestyle{fancy}

% testo
\subsection{Introduzione}

\subsubsection{Programma del corso}
Il corso di ricerca operativa si divide in 4 parti:

\begin{enumerate}
	\item Modello di Programmazione Lineare;
	\item Programmazione Lineare su reti, ergo programmazione lineare su grafi;
	\item Programmazione Lineare intera, ergo programmazione lineare col vincolo $x \in \mathbb{Z}^n$;
	\item Programmazione Non Lineare.
\end{enumerate}

Le prime 3 parti hanno come prerequisiti l'algebra lineare: in particolare operazioni matriciali, prodotti scalari, sistemi lineari, teorema di Rouché-Capelli.
La quarta parte richiede invece conoscenze di Analisi II.

\subsubsection{Un problema di programmazione lineare}

La ricerca operativa si occupa di risolvere problemi di ottimizzazione con variabili decisionali e risorse limitate.
Poniamo un problema di esempio:

\begin{problem}{Produzione}
Una ditta produce due prodotti: \textbf{laminato A} e \textbf{laminato B}.
Ogni prodotto deve passare attraverso diversi reparti: il reparto \textbf{materie prime}, il reparto \textbf{taglio}, il reparto \textbf{finiture A} e il reparto \textbf{finiture B}.
Il guadagno è rispettivamente di 8.4 e 11.2 (unità di misura irrilevante) per ogni tipo di laminato.

Ora, nel reparto materie prime, il laminato A occupa 30, ore, e lo B 20 ore.
Nel reparto taglio il laminato A occupa 10 ore e lo B 20 ore.
Il laminato A occupa poi 20 ore nel reparto finiture A, mentre il laminato B occupa 30 ore nel reparto finiture B.
I reparti hanno a disposizione, rispettivamente, 120, 80, 62 e 105 ore.
Possiamo porre queste informazioni in forma tabulare:

	\center \rowcolors{2}{green!10}{green!40!black!20}
	\begin{tabular} { | c || c | c | c | }
		\hline
		\bfseries Reparto & \bfseries Capienza & \bfseries Laminato A & \bfseries Laminato B \\
		\hline 
		Materie prime & 120 & 30 & 20 \\
		Taglio & 80 & 10 & 20 \\
		Finiture A & 62 & 20 & / \\
		Finiture B & 105 & / & 30 \\
		\hline
		\textbf{Guadagno} & & 8.4 & 11.2 \\
		\hline
	\end{tabular}

	\par\bigskip

Quello che ci interessa è chiaramente massimizzare il guadagno.
\end{problem}

Decidiamo di modellizzare questa situazione con un modello matematico.

Il guadagno che abbiamo dai laminati rappresenta una \textbf{funzione obiettivo}, ovvero la funzione che vogliamo ottimizzare.
Ottimizzare significa trovare il modo migliore di massimizzare o minimizzare i valori della funzione agendo sulle variabili decisionali.
La funzione obiettivo va ottimizzata rispettando determinati \textbf{vincoli}, che modellizzano il fatto che le risorse sono limitate.
Una \textbf{soluzione ammissibile} è una qualsiasi soluzione che rispetta i vincoli del problema.
Chiamiamo quindi \textbf{regione ammissibile} l'insieme di tutte le soluzioni ammissibili.
All'interno della regione ammissibile c'è la soluzione che cerchiamo, ovvero la \textbf{soluzione ottima}.

Decidiamo quindi le \textbf{variabili decisionali}, ed esplicitiamo la funzione obiettivo e i vincoli.

In questo caso le variabili decisionali saranno le quantità di laminato A e B da produrre, che individuano un punto in $ \mathbb{R}^2 $ denominato $ ( x_A, x_B ) $. 
Decidere di usare la soluzione $ (1,1) $ significa decidere di produrre 1 unità di laminato A e 1 unità di laminato B, per un guadagno complessivo di $ 8.4 + 11.2 = 19.6 $.

La funzione obiettivo sarà quindi:

$$ f(x_A, x_B) = 8.4 x_A + 11.2 x_B, \quad f: \mathbb{R}^2 \rightarrow \mathbb{R} $$

lineare, e noi saremo interessati a:

$$ \max(f(x_A, x_B)) $$

rispettando i vincoli, ergo nella regione ammissibile.
Per esprimere questi vincoli, cioè il tempo limitato all'interno di ogni reparto, introduciamo il sistema di disequazioni:

\[
	\begin{cases}
		30 x_A + 20 x_B \leq 120 \\
		10 x_A + 20 x_B \leq 80	\\
		20 x_A + 0 x_B \leq 62 \\	
		0 x_A + 30 x_B \leq 105 \\
		- x_A \leq 0 \\
		- x_B \leq 0 \\
	\end{cases}
\]

dove notiamo le ultime due disequazioni indicano la positività di $x_A$ e $x_B$, in forma $ f(x_A, x_B) \leq b $.
Questo sistema non indica altro che la regione ammissibile.

Possiamo riscrivere questo modello usando la notazione dell'algebra lineare.
La funzione obiettiva e i vincoli diventano semplicemente:

\[
	\begin{cases}
		\max(c^T \cdot x) \\
		A \cdot x \leq b	
	\end{cases}
\]

dove $c$ rappresenta il vettore dei costi, $A$ rappresenta la matrice dei costi a $b$ il vettore dei vincoli.
$c$ è trasposto per indicare prodotto fra vettori.

Possiamo scrivere $A$, $b$ e $c$ per esteso:

$$
A:
\begin{pmatrix}
	30 & 20 \\
	10 & 20 \\
	20 & 0 \\
	0 & 30 \\
	-1 & 0 \\
	0 & -1
\end{pmatrix}, \quad
b:
\begin{pmatrix}
	120 \\
	80 \\
	62 \\ 
	105 \\ 
	0 \\ 
	0 
\end{pmatrix}, \quad 
c:
\begin{pmatrix}
	8.4 \\
	11.2 \\
\end{pmatrix}
$$

Notiamo come $A$ e $b$ hanno dimensione verticale $ 4 + 2 = 6 $, dai 4 vincoli superiori e i 2 vincoli inferiori.

A questo punto, possiamo disegnare la regione ammissibile come l'intersezione dei semipiani individuati da ogni singola disuguaglianza.

Si riporta un grafico:

\begin{center}

\begin{tikzpicture}
\begin{axis}[
    axis lines = middle,
    xlabel = {$x_A$},
    ylabel = {$x_B$},
    xmin=0, xmax=7.9,
    ymin=0, ymax=3.9,
    domain=0:10,
    samples=100,
    width=15cm, height=7cm,
    legend pos=north east
  ]

% regione ammissibile

	\addplot[fill=gray, opacity=0.4, forget plot] 
    coordinates {
			(0, 0)
			(3.1, 0)
			(3.1, 1.35)
			(2,3)
			(1, 3.5)
			(0, 3.5)
		};

% rette

\addplot[domain=2:3.1, thick, blue] {6 - 1.5*x}; 
\addlegendentry{$ 30 x_A + 20 x_B \leq 120 $}

\addplot[domain=1:2, thick, green] {4 - 0.5*x}; 
\addlegendentry{$ 10 x_A + 20 x_B \leq 80 $}

\addplot[thick, purple] coordinates {(3.1, 0) (3.1, 1.35)};
\addlegendentry{$ 20 x_A + 0 x_B \leq 62 $}

\addplot[domain=0:1, thick, red] {3.5}; 
\addlegendentry{$ 0 x_A + 30 x_B \leq 105 $}
	
\end{axis}
\end{tikzpicture}

\end{center}

In diversi colori sono riportati i margini delle disequazioni, mentre in grigio è evidenziata la regione ammissibile.
Qualsiasi punto all'interno della regione ammissibile vale come soluzione, e almeno uno di essi è soluzione ottimale.

\par\smallskip

Il modello finora descritto prende il nome di modello di programmazione lineare, e permette di formulare problemi di programmazione lineare (LP).

\begin{definition}{Problema di programmazione lineare (1)}
Un problema di programmazione lineare (LP) riguarda l'ottimizzazione di una funzione lineare in più variabili
soggetta a vincoli di tipo $ =, \ \leq $ e $ \geq $, ovvero in forma:
\[
	\begin{cases}
			\min / \max(c^T \cdot x) \\
			A_i x \leq b \\
			B_j x \geq d \\
			C_k x = e \\
	\end{cases}
\]
\end{definition}

"Programmazione" qui non ha alcun legame col concetto di programmazione informatica, ma si riferisce al fatto che il modello è effettivamente \textit{programmabile}.

"Lineare" si riferisce alla linearità dei problemi che ci permette di risolvere (e quindi del modello).




\end{document}

\documentclass[a4paper,11pt]{article}
\usepackage[a4paper, margin=8em]{geometry}

% usa i pacchetti per la scrittura in italiano
\usepackage[french,italian]{babel}
\usepackage[T1]{fontenc}
\usepackage[utf8]{inputenc}
\frenchspacing 

% usa i pacchetti per la formattazione matematica
\usepackage{amsmath, amssymb, amsthm, amsfonts}

% usa altri pacchetti
\usepackage{gensymb}
\usepackage{hyperref}
\usepackage{standalone}

% imposta il titolo
\title{Appunti Ricerca Operativa}
\author{Luca Seggiani}
\date{23-09-24}

% imposta lo stile
% usa helvetica
\usepackage[scaled]{helvet}
% usa palatino
\usepackage{palatino}
% usa un font monospazio guardabile
\usepackage{lmodern}

\renewcommand{\rmdefault}{ppl}
\renewcommand{\sfdefault}{phv}
\renewcommand{\ttdefault}{lmtt}

% disponi teoremi
\usepackage{tcolorbox}
\newtcolorbox[auto counter, number within=section]{theorem}[2][]{%
	colback=blue!10, 
	colframe=blue!40!black, 
	sharp corners=northwest,
	fonttitle=\sffamily\bfseries, 
	title=Teorema~\thetcbcounter: #2, 
	#1
}

% disponi definizioni
\newtcolorbox[auto counter, number within=section]{definition}[2][]{%
	colback=red!10,
	colframe=red!40!black,
	sharp corners=northwest,
	fonttitle=\sffamily\bfseries,
	title=Definizione~\thetcbcounter: #2,
	#1
}

% disponi problemi
\newtcolorbox[auto counter, number within=section]{problem}[2][]{%
	colback=green!10,
	colframe=green!40!black,
	sharp corners=northwest,
	fonttitle=\sffamily\bfseries,
	title=Problema~\thetcbcounter: #2,
	#1
}

% disponi codice
\usepackage{listings}
\usepackage[table]{xcolor}

\lstdefinestyle{codestyle}{
		backgroundcolor=\color{black!5}, 
		commentstyle=\color{codegreen},
		keywordstyle=\bfseries\color{magenta},
		numberstyle=\sffamily\tiny\color{black!60},
		stringstyle=\color{green!50!black},
		basicstyle=\ttfamily\footnotesize,
		breakatwhitespace=false,         
		breaklines=true,                 
		captionpos=b,                    
		keepspaces=true,                 
		numbers=left,                    
		numbersep=5pt,                  
		showspaces=false,                
		showstringspaces=false,
		showtabs=false,                  
		tabsize=2
}

\lstdefinestyle{shellstyle}{
		backgroundcolor=\color{black!5}, 
		basicstyle=\ttfamily\footnotesize\color{black}, 
		commentstyle=\color{black}, 
		keywordstyle=\color{black},
		numberstyle=\color{black!5},
		stringstyle=\color{black}, 
		showspaces=false,
		showstringspaces=false, 
		showtabs=false, 
		tabsize=2, 
		numbers=none, 
		breaklines=true
}

\lstdefinelanguage{javascript}{
	keywords={typeof, new, true, false, catch, function, return, null, catch, switch, var, if, in, while, do, else, case, break},
	keywordstyle=\color{blue}\bfseries,
	ndkeywords={class, export, boolean, throw, implements, import, this},
	ndkeywordstyle=\color{darkgray}\bfseries,
	identifierstyle=\color{black},
	sensitive=false,
	comment=[l]{//},
	morecomment=[s]{/*}{*/},
	commentstyle=\color{purple}\ttfamily,
	stringstyle=\color{red}\ttfamily,
	morestring=[b]',
	morestring=[b]"
}

% disponi sezioni
\usepackage{titlesec}

\titleformat{\section}
	{\sffamily\Large\bfseries} 
	{\thesection}{1em}{} 
\titleformat{\subsection}
	{\sffamily\large\bfseries}   
	{\thesubsection}{1em}{} 
\titleformat{\subsubsection}
	{\sffamily\normalsize\bfseries} 
	{\thesubsubsection}{1em}{}

% disponi alberi
\usepackage{forest}

\forestset{
	rectstyle/.style={
		for tree={rectangle,draw,font=\large\sffamily}
	},
	roundstyle/.style={
		for tree={circle,draw,font=\large}
	}
}

% disponi algoritmi
\usepackage{algorithm}
\usepackage{algorithmic}
\makeatletter
\renewcommand{\ALG@name}{Algoritmo}
\makeatother

% disponi numeri di pagina
\usepackage{fancyhdr}
\fancyhf{} 
\fancyfoot[L]{\sffamily{\thepage}}

\makeatletter
\fancyhead[L]{\raisebox{1ex}[0pt][0pt]{\sffamily{\@title \ \@date}}} 
\fancyhead[R]{\raisebox{1ex}[0pt][0pt]{\sffamily{\@author}}}
\makeatother

% disegni
\usepackage{pgfplots}
\pgfplotsset{width=10cm,compat=1.9}

\begin{document}
% sezione (data)
\section{Lezione del 23-09-24}

% stili pagina
\thispagestyle{empty}
\pagestyle{fancy}

% testo
\subsection{Introduzione}

\subsubsection{Programma del corso}
Il corso di ricerca operativa si divide in 4 parti:

\begin{enumerate}
	\item Modello di Programmazione Lineare;
	\item Programmazione Lineare su reti, ergo programmazione lineare su grafi;
	\item Programmazione Lineare intera, ergo programmazione lineare col vincolo $x \in \mathbb{Z}^n$;
	\item Programmazione Non Lineare.
\end{enumerate}

Le prime 3 parti hanno come prerequisiti l'algebra lineare: in particolare operazioni matriciali, prodotti scalari, sistemi lineari, teorema di Rouché-Capelli.
La quarta parte richiede invece conoscenze di Analisi II.

\subsubsection{Un problema di programmazione lineare}

La ricerca operativa si occupa di risolvere problemi di ottimizzazione con variabili decisionali e risorse limitate.
Poniamo un problema di esempio:

\begin{problem}{Produzione}
Una ditta produce due prodotti: \textbf{laminato A} e \textbf{laminato B}.
Ogni prodotto deve passare attraverso diversi reparti: il reparto \textbf{materie prime}, il reparto \textbf{taglio}, il reparto \textbf{finiture A} e il reparto \textbf{finiture B}.
Il guadagno è rispettivamente di 8.4 e 11.2 (unità di misura irrilevante) per ogni tipo di laminato.

Ora, nel reparto materie prime, il laminato A occupa 30, ore, e lo B 20 ore.
Nel reparto taglio il laminato A occupa 10 ore e lo B 20 ore.
Il laminato A occupa poi 20 ore nel reparto finiture A, mentre il laminato B occupa 30 ore nel reparto finiture B.
I reparti hanno a disposizione, rispettivamente, 120, 80, 62 e 105 ore.
Possiamo porre queste informazioni in forma tabulare:

	\center \rowcolors{2}{green!10}{green!40!black!20}
	\begin{tabular} { | c || c | c | c | }
		\hline
		\bfseries Reparto & \bfseries Capienza & \bfseries Laminato A & \bfseries Laminato B \\
		\hline 
		Materie prime & 120 & 30 & 20 \\
		Taglio & 80 & 10 & 20 \\
		Finiture A & 62 & 20 & / \\
		Finiture B & 105 & / & 30 \\
		\hline
		\textbf{Guadagno} & & 8.4 & 11.2 \\
		\hline
	\end{tabular}

	\par\bigskip

Quello che ci interessa è chiaramente massimizzare il guadagno.
\end{problem}

Decidiamo di modellizzare questa situazione con un modello matematico.

Il guadagno che abbiamo dai laminati rappresenta una \textbf{funzione obiettivo}, ovvero la funzione che vogliamo ottimizzare.
Ottimizzare significa trovare il modo migliore di massimizzare o minimizzare i valori della funzione agendo sulle variabili decisionali.
La funzione obiettivo va ottimizzata rispettando determinati \textbf{vincoli}, che modellizzano il fatto che le risorse sono limitate.
Una \textbf{soluzione ammissibile} è una qualsiasi soluzione che rispetta i vincoli del problema.
Chiamiamo quindi \textbf{regione ammissibile} l'insieme di tutte le soluzioni ammissibili.
All'interno della regione ammissibile c'è la soluzione che cerchiamo, ovvero la \textbf{soluzione ottima}.

Decidiamo quindi le \textbf{variabili decisionali}, ed esplicitiamo la funzione obiettivo e i vincoli.

In questo caso le variabili decisionali saranno le quantità di laminato A e B da produrre, che individuano un punto in $ \mathbb{R}^2 $ denominato $ ( x_A, x_B ) $. 
Decidere di usare la soluzione $ (1,1) $ significa decidere di produrre 1 unità di laminato A e 1 unità di laminato B, per un guadagno complessivo di $ 8.4 + 11.2 = 19.6 $.

La funzione obiettivo sarà quindi:

$$ f(x_A, x_B) = 8.4 x_A + 11.2 x_B, \quad f: \mathbb{R}^2 \rightarrow \mathbb{R} $$

lineare, e noi saremo interessati a:

$$ \max(f(x_A, x_B)) $$

rispettando i vincoli, ergo nella regione ammissibile.
Per esprimere questi vincoli, cioè il tempo limitato all'interno di ogni reparto, introduciamo il sistema di disequazioni:

\[
	\begin{cases}
		30 x_A + 20 x_B \leq 120 \\
		10 x_A + 20 x_B \leq 80	\\
		20 x_A + 0 x_B \leq 62 \\	
		0 x_A + 30 x_B \leq 105 \\
		- x_A \leq 0 \\
		- x_B \leq 0 \\
	\end{cases}
\]

dove notiamo le ultime due disequazioni indicano la positività di $x_A$ e $x_B$, in forma $ f(x_A, x_B) \leq b $.
Questo sistema non indica altro che la regione ammissibile.

Possiamo riscrivere questo modello usando la notazione dell'algebra lineare.
La funzione obiettiva e i vincoli diventano semplicemente:

\[
	\begin{cases}
		\max(c^T \cdot x) \\
		A \cdot x \leq b	
	\end{cases}
\]

dove $c$ rappresenta il vettore dei costi, $A$ rappresenta la matrice dei costi a $b$ il vettore dei vincoli.
$c$ è trasposto per indicare prodotto fra vettori.

Possiamo scrivere $A$, $b$ e $c$ per esteso:

$$
A:
\begin{pmatrix}
	30 & 20 \\
	10 & 20 \\
	20 & 0 \\
	0 & 30 \\
	-1 & 0 \\
	0 & -1
\end{pmatrix}, \quad
b:
\begin{pmatrix}
	120 \\
	80 \\
	62 \\ 
	105 \\ 
	0 \\ 
	0 
\end{pmatrix}, \quad 
c:
\begin{pmatrix}
	8.4 \\
	11.2 \\
\end{pmatrix}
$$

Notiamo come $A$ e $b$ hanno dimensione verticale $ 4 + 2 = 6 $, dai 4 vincoli superiori e i 2 vincoli inferiori.

A questo punto, possiamo disegnare la regione ammissibile come l'intersezione dei semipiani individuati da ogni singola disuguaglianza.

Si riporta un grafico:

\begin{center}

\begin{tikzpicture}
\begin{axis}[
    axis lines = middle,
    xlabel = {$x_A$},
    ylabel = {$x_B$},
    xmin=0, xmax=7.9,
    ymin=0, ymax=3.9,
    domain=0:10,
    samples=100,
    width=15cm, height=7cm,
    legend pos=north east
  ]

% regione ammissibile

	\addplot[fill=gray, opacity=0.4, forget plot] 
    coordinates {
			(0, 0)
			(3.1, 0)
			(3.1, 1.35)
			(2,3)
			(1, 3.5)
			(0, 3.5)
		};

% rette

\addplot[domain=2:3.1, thick, blue] {6 - 1.5*x}; 
\addlegendentry{$ 30 x_A + 20 x_B \leq 120 $}

\addplot[domain=1:2, thick, green] {4 - 0.5*x}; 
\addlegendentry{$ 10 x_A + 20 x_B \leq 80 $}

\addplot[thick, purple] coordinates {(3.1, 0) (3.1, 1.35)};
\addlegendentry{$ 20 x_A + 0 x_B \leq 62 $}

\addplot[domain=0:1, thick, red] {3.5}; 
\addlegendentry{$ 0 x_A + 30 x_B \leq 105 $}
	
\end{axis}
\end{tikzpicture}

\end{center}

In diversi colori sono riportati i margini delle disequazioni, mentre in grigio è evidenziata la regione ammissibile.
Qualsiasi punto all'interno della regione ammissibile vale come soluzione, e almeno uno di essi è soluzione ottimale.

\par\smallskip

Il modello finora descritto prende il nome di modello di programmazione lineare, e permette di formulare problemi di programmazione lineare (LP).

\begin{definition}{Problema di programmazione lineare (1)}
Un problema di programmazione lineare (LP) riguarda l'ottimizzazione di una funzione lineare in più variabili
soggetta a vincoli di tipo $ =, \ \leq $ e $ \geq $, ovvero in forma:
\[
	\begin{cases}
			\min / \max(c^T \cdot x) \\
			A_i x \leq b \\
			B_j x \geq d \\
			C_k x = e \\
	\end{cases}
\]
\end{definition}

"Programmazione" qui non ha alcun legame col concetto di programmazione informatica, ma si riferisce al fatto che il modello è effettivamente \textit{programmabile}.

"Lineare" si riferisce alla linearità dei problemi che ci permette di risolvere (e quindi del modello).




\end{document}

\documentclass[a4paper,11pt]{article}
\usepackage[a4paper, margin=8em]{geometry}

% usa i pacchetti per la scrittura in italiano
\usepackage[french,italian]{babel}
\usepackage[T1]{fontenc}
\usepackage[utf8]{inputenc}
\frenchspacing 

% usa i pacchetti per la formattazione matematica
\usepackage{amsmath, amssymb, amsthm, amsfonts}

% usa altri pacchetti
\usepackage{gensymb}
\usepackage{hyperref}
\usepackage{standalone}

% imposta il titolo
\title{Appunti Ricerca Operativa}
\author{Luca Seggiani}
\date{23-09-24}

% imposta lo stile
% usa helvetica
\usepackage[scaled]{helvet}
% usa palatino
\usepackage{palatino}
% usa un font monospazio guardabile
\usepackage{lmodern}

\renewcommand{\rmdefault}{ppl}
\renewcommand{\sfdefault}{phv}
\renewcommand{\ttdefault}{lmtt}

% disponi teoremi
\usepackage{tcolorbox}
\newtcolorbox[auto counter, number within=section]{theorem}[2][]{%
	colback=blue!10, 
	colframe=blue!40!black, 
	sharp corners=northwest,
	fonttitle=\sffamily\bfseries, 
	title=Teorema~\thetcbcounter: #2, 
	#1
}

% disponi definizioni
\newtcolorbox[auto counter, number within=section]{definition}[2][]{%
	colback=red!10,
	colframe=red!40!black,
	sharp corners=northwest,
	fonttitle=\sffamily\bfseries,
	title=Definizione~\thetcbcounter: #2,
	#1
}

% disponi problemi
\newtcolorbox[auto counter, number within=section]{problem}[2][]{%
	colback=green!10,
	colframe=green!40!black,
	sharp corners=northwest,
	fonttitle=\sffamily\bfseries,
	title=Problema~\thetcbcounter: #2,
	#1
}

% disponi codice
\usepackage{listings}
\usepackage[table]{xcolor}

\lstdefinestyle{codestyle}{
		backgroundcolor=\color{black!5}, 
		commentstyle=\color{codegreen},
		keywordstyle=\bfseries\color{magenta},
		numberstyle=\sffamily\tiny\color{black!60},
		stringstyle=\color{green!50!black},
		basicstyle=\ttfamily\footnotesize,
		breakatwhitespace=false,         
		breaklines=true,                 
		captionpos=b,                    
		keepspaces=true,                 
		numbers=left,                    
		numbersep=5pt,                  
		showspaces=false,                
		showstringspaces=false,
		showtabs=false,                  
		tabsize=2
}

\lstdefinestyle{shellstyle}{
		backgroundcolor=\color{black!5}, 
		basicstyle=\ttfamily\footnotesize\color{black}, 
		commentstyle=\color{black}, 
		keywordstyle=\color{black},
		numberstyle=\color{black!5},
		stringstyle=\color{black}, 
		showspaces=false,
		showstringspaces=false, 
		showtabs=false, 
		tabsize=2, 
		numbers=none, 
		breaklines=true
}

\lstdefinelanguage{javascript}{
	keywords={typeof, new, true, false, catch, function, return, null, catch, switch, var, if, in, while, do, else, case, break},
	keywordstyle=\color{blue}\bfseries,
	ndkeywords={class, export, boolean, throw, implements, import, this},
	ndkeywordstyle=\color{darkgray}\bfseries,
	identifierstyle=\color{black},
	sensitive=false,
	comment=[l]{//},
	morecomment=[s]{/*}{*/},
	commentstyle=\color{purple}\ttfamily,
	stringstyle=\color{red}\ttfamily,
	morestring=[b]',
	morestring=[b]"
}

% disponi sezioni
\usepackage{titlesec}

\titleformat{\section}
	{\sffamily\Large\bfseries} 
	{\thesection}{1em}{} 
\titleformat{\subsection}
	{\sffamily\large\bfseries}   
	{\thesubsection}{1em}{} 
\titleformat{\subsubsection}
	{\sffamily\normalsize\bfseries} 
	{\thesubsubsection}{1em}{}

% disponi alberi
\usepackage{forest}

\forestset{
	rectstyle/.style={
		for tree={rectangle,draw,font=\large\sffamily}
	},
	roundstyle/.style={
		for tree={circle,draw,font=\large}
	}
}

% disponi algoritmi
\usepackage{algorithm}
\usepackage{algorithmic}
\makeatletter
\renewcommand{\ALG@name}{Algoritmo}
\makeatother

% disponi numeri di pagina
\usepackage{fancyhdr}
\fancyhf{} 
\fancyfoot[L]{\sffamily{\thepage}}

\makeatletter
\fancyhead[L]{\raisebox{1ex}[0pt][0pt]{\sffamily{\@title \ \@date}}} 
\fancyhead[R]{\raisebox{1ex}[0pt][0pt]{\sffamily{\@author}}}
\makeatother

% disegni
\usepackage{pgfplots}
\pgfplotsset{width=10cm,compat=1.9}

\begin{document}
% sezione (data)
\section{Lezione del 23-09-24}

% stili pagina
\thispagestyle{empty}
\pagestyle{fancy}

% testo
\subsection{Introduzione}

\subsubsection{Programma del corso}
Il corso di ricerca operativa si divide in 4 parti:

\begin{enumerate}
	\item Modello di Programmazione Lineare;
	\item Programmazione Lineare su reti, ergo programmazione lineare su grafi;
	\item Programmazione Lineare intera, ergo programmazione lineare col vincolo $x \in \mathbb{Z}^n$;
	\item Programmazione Non Lineare.
\end{enumerate}

Le prime 3 parti hanno come prerequisiti l'algebra lineare: in particolare operazioni matriciali, prodotti scalari, sistemi lineari, teorema di Rouché-Capelli.
La quarta parte richiede invece conoscenze di Analisi II.

\subsubsection{Un problema di programmazione lineare}

La ricerca operativa si occupa di risolvere problemi di ottimizzazione con variabili decisionali e risorse limitate.
Poniamo un problema di esempio:

\begin{problem}{Produzione}
Una ditta produce due prodotti: \textbf{laminato A} e \textbf{laminato B}.
Ogni prodotto deve passare attraverso diversi reparti: il reparto \textbf{materie prime}, il reparto \textbf{taglio}, il reparto \textbf{finiture A} e il reparto \textbf{finiture B}.
Il guadagno è rispettivamente di 8.4 e 11.2 (unità di misura irrilevante) per ogni tipo di laminato.

Ora, nel reparto materie prime, il laminato A occupa 30, ore, e lo B 20 ore.
Nel reparto taglio il laminato A occupa 10 ore e lo B 20 ore.
Il laminato A occupa poi 20 ore nel reparto finiture A, mentre il laminato B occupa 30 ore nel reparto finiture B.
I reparti hanno a disposizione, rispettivamente, 120, 80, 62 e 105 ore.
Possiamo porre queste informazioni in forma tabulare:

	\center \rowcolors{2}{green!10}{green!40!black!20}
	\begin{tabular} { | c || c | c | c | }
		\hline
		\bfseries Reparto & \bfseries Capienza & \bfseries Laminato A & \bfseries Laminato B \\
		\hline 
		Materie prime & 120 & 30 & 20 \\
		Taglio & 80 & 10 & 20 \\
		Finiture A & 62 & 20 & / \\
		Finiture B & 105 & / & 30 \\
		\hline
		\textbf{Guadagno} & & 8.4 & 11.2 \\
		\hline
	\end{tabular}

	\par\bigskip

Quello che ci interessa è chiaramente massimizzare il guadagno.
\end{problem}

Decidiamo di modellizzare questa situazione con un modello matematico.

Il guadagno che abbiamo dai laminati rappresenta una \textbf{funzione obiettivo}, ovvero la funzione che vogliamo ottimizzare.
Ottimizzare significa trovare il modo migliore di massimizzare o minimizzare i valori della funzione agendo sulle variabili decisionali.
La funzione obiettivo va ottimizzata rispettando determinati \textbf{vincoli}, che modellizzano il fatto che le risorse sono limitate.
Una \textbf{soluzione ammissibile} è una qualsiasi soluzione che rispetta i vincoli del problema.
Chiamiamo quindi \textbf{regione ammissibile} l'insieme di tutte le soluzioni ammissibili.
All'interno della regione ammissibile c'è la soluzione che cerchiamo, ovvero la \textbf{soluzione ottima}.

Decidiamo quindi le \textbf{variabili decisionali}, ed esplicitiamo la funzione obiettivo e i vincoli.

In questo caso le variabili decisionali saranno le quantità di laminato A e B da produrre, che individuano un punto in $ \mathbb{R}^2 $ denominato $ ( x_A, x_B ) $. 
Decidere di usare la soluzione $ (1,1) $ significa decidere di produrre 1 unità di laminato A e 1 unità di laminato B, per un guadagno complessivo di $ 8.4 + 11.2 = 19.6 $.

La funzione obiettivo sarà quindi:

$$ f(x_A, x_B) = 8.4 x_A + 11.2 x_B, \quad f: \mathbb{R}^2 \rightarrow \mathbb{R} $$

lineare, e noi saremo interessati a:

$$ \max(f(x_A, x_B)) $$

rispettando i vincoli, ergo nella regione ammissibile.
Per esprimere questi vincoli, cioè il tempo limitato all'interno di ogni reparto, introduciamo il sistema di disequazioni:

\[
	\begin{cases}
		30 x_A + 20 x_B \leq 120 \\
		10 x_A + 20 x_B \leq 80	\\
		20 x_A + 0 x_B \leq 62 \\	
		0 x_A + 30 x_B \leq 105 \\
		- x_A \leq 0 \\
		- x_B \leq 0 \\
	\end{cases}
\]

dove notiamo le ultime due disequazioni indicano la positività di $x_A$ e $x_B$, in forma $ f(x_A, x_B) \leq b $.
Questo sistema non indica altro che la regione ammissibile.

Possiamo riscrivere questo modello usando la notazione dell'algebra lineare.
La funzione obiettiva e i vincoli diventano semplicemente:

\[
	\begin{cases}
		\max(c^T \cdot x) \\
		A \cdot x \leq b	
	\end{cases}
\]

dove $c$ rappresenta il vettore dei costi, $A$ rappresenta la matrice dei costi a $b$ il vettore dei vincoli.
$c$ è trasposto per indicare prodotto fra vettori.

Possiamo scrivere $A$, $b$ e $c$ per esteso:

$$
A:
\begin{pmatrix}
	30 & 20 \\
	10 & 20 \\
	20 & 0 \\
	0 & 30 \\
	-1 & 0 \\
	0 & -1
\end{pmatrix}, \quad
b:
\begin{pmatrix}
	120 \\
	80 \\
	62 \\ 
	105 \\ 
	0 \\ 
	0 
\end{pmatrix}, \quad 
c:
\begin{pmatrix}
	8.4 \\
	11.2 \\
\end{pmatrix}
$$

Notiamo come $A$ e $b$ hanno dimensione verticale $ 4 + 2 = 6 $, dai 4 vincoli superiori e i 2 vincoli inferiori.

A questo punto, possiamo disegnare la regione ammissibile come l'intersezione dei semipiani individuati da ogni singola disuguaglianza.

Si riporta un grafico:

\begin{center}

\begin{tikzpicture}
\begin{axis}[
    axis lines = middle,
    xlabel = {$x_A$},
    ylabel = {$x_B$},
    xmin=0, xmax=7.9,
    ymin=0, ymax=3.9,
    domain=0:10,
    samples=100,
    width=15cm, height=7cm,
    legend pos=north east
  ]

% regione ammissibile

	\addplot[fill=gray, opacity=0.4, forget plot] 
    coordinates {
			(0, 0)
			(3.1, 0)
			(3.1, 1.35)
			(2,3)
			(1, 3.5)
			(0, 3.5)
		};

% rette

\addplot[domain=2:3.1, thick, blue] {6 - 1.5*x}; 
\addlegendentry{$ 30 x_A + 20 x_B \leq 120 $}

\addplot[domain=1:2, thick, green] {4 - 0.5*x}; 
\addlegendentry{$ 10 x_A + 20 x_B \leq 80 $}

\addplot[thick, purple] coordinates {(3.1, 0) (3.1, 1.35)};
\addlegendentry{$ 20 x_A + 0 x_B \leq 62 $}

\addplot[domain=0:1, thick, red] {3.5}; 
\addlegendentry{$ 0 x_A + 30 x_B \leq 105 $}
	
\end{axis}
\end{tikzpicture}

\end{center}

In diversi colori sono riportati i margini delle disequazioni, mentre in grigio è evidenziata la regione ammissibile.
Qualsiasi punto all'interno della regione ammissibile vale come soluzione, e almeno uno di essi è soluzione ottimale.

\par\smallskip

Il modello finora descritto prende il nome di modello di programmazione lineare, e permette di formulare problemi di programmazione lineare (LP).

\begin{definition}{Problema di programmazione lineare (1)}
Un problema di programmazione lineare (LP) riguarda l'ottimizzazione di una funzione lineare in più variabili
soggetta a vincoli di tipo $ =, \ \leq $ e $ \geq $, ovvero in forma:
\[
	\begin{cases}
			\min / \max(c^T \cdot x) \\
			A_i x \leq b \\
			B_j x \geq d \\
			C_k x = e \\
	\end{cases}
\]
\end{definition}

"Programmazione" qui non ha alcun legame col concetto di programmazione informatica, ma si riferisce al fatto che il modello è effettivamente \textit{programmabile}.

"Lineare" si riferisce alla linearità dei problemi che ci permette di risolvere (e quindi del modello).




\end{document}

\documentclass[a4paper,11pt]{article}
\usepackage[a4paper, margin=8em]{geometry}

% usa i pacchetti per la scrittura in italiano
\usepackage[french,italian]{babel}
\usepackage[T1]{fontenc}
\usepackage[utf8]{inputenc}
\frenchspacing 

% usa i pacchetti per la formattazione matematica
\usepackage{amsmath, amssymb, amsthm, amsfonts}

% usa altri pacchetti
\usepackage{gensymb}
\usepackage{hyperref}
\usepackage{standalone}

% imposta il titolo
\title{Appunti Ricerca Operativa}
\author{Luca Seggiani}
\date{23-09-24}

% imposta lo stile
% usa helvetica
\usepackage[scaled]{helvet}
% usa palatino
\usepackage{palatino}
% usa un font monospazio guardabile
\usepackage{lmodern}

\renewcommand{\rmdefault}{ppl}
\renewcommand{\sfdefault}{phv}
\renewcommand{\ttdefault}{lmtt}

% disponi teoremi
\usepackage{tcolorbox}
\newtcolorbox[auto counter, number within=section]{theorem}[2][]{%
	colback=blue!10, 
	colframe=blue!40!black, 
	sharp corners=northwest,
	fonttitle=\sffamily\bfseries, 
	title=Teorema~\thetcbcounter: #2, 
	#1
}

% disponi definizioni
\newtcolorbox[auto counter, number within=section]{definition}[2][]{%
	colback=red!10,
	colframe=red!40!black,
	sharp corners=northwest,
	fonttitle=\sffamily\bfseries,
	title=Definizione~\thetcbcounter: #2,
	#1
}

% disponi problemi
\newtcolorbox[auto counter, number within=section]{problem}[2][]{%
	colback=green!10,
	colframe=green!40!black,
	sharp corners=northwest,
	fonttitle=\sffamily\bfseries,
	title=Problema~\thetcbcounter: #2,
	#1
}

% disponi codice
\usepackage{listings}
\usepackage[table]{xcolor}

\lstdefinestyle{codestyle}{
		backgroundcolor=\color{black!5}, 
		commentstyle=\color{codegreen},
		keywordstyle=\bfseries\color{magenta},
		numberstyle=\sffamily\tiny\color{black!60},
		stringstyle=\color{green!50!black},
		basicstyle=\ttfamily\footnotesize,
		breakatwhitespace=false,         
		breaklines=true,                 
		captionpos=b,                    
		keepspaces=true,                 
		numbers=left,                    
		numbersep=5pt,                  
		showspaces=false,                
		showstringspaces=false,
		showtabs=false,                  
		tabsize=2
}

\lstdefinestyle{shellstyle}{
		backgroundcolor=\color{black!5}, 
		basicstyle=\ttfamily\footnotesize\color{black}, 
		commentstyle=\color{black}, 
		keywordstyle=\color{black},
		numberstyle=\color{black!5},
		stringstyle=\color{black}, 
		showspaces=false,
		showstringspaces=false, 
		showtabs=false, 
		tabsize=2, 
		numbers=none, 
		breaklines=true
}

\lstdefinelanguage{javascript}{
	keywords={typeof, new, true, false, catch, function, return, null, catch, switch, var, if, in, while, do, else, case, break},
	keywordstyle=\color{blue}\bfseries,
	ndkeywords={class, export, boolean, throw, implements, import, this},
	ndkeywordstyle=\color{darkgray}\bfseries,
	identifierstyle=\color{black},
	sensitive=false,
	comment=[l]{//},
	morecomment=[s]{/*}{*/},
	commentstyle=\color{purple}\ttfamily,
	stringstyle=\color{red}\ttfamily,
	morestring=[b]',
	morestring=[b]"
}

% disponi sezioni
\usepackage{titlesec}

\titleformat{\section}
	{\sffamily\Large\bfseries} 
	{\thesection}{1em}{} 
\titleformat{\subsection}
	{\sffamily\large\bfseries}   
	{\thesubsection}{1em}{} 
\titleformat{\subsubsection}
	{\sffamily\normalsize\bfseries} 
	{\thesubsubsection}{1em}{}

% disponi alberi
\usepackage{forest}

\forestset{
	rectstyle/.style={
		for tree={rectangle,draw,font=\large\sffamily}
	},
	roundstyle/.style={
		for tree={circle,draw,font=\large}
	}
}

% disponi algoritmi
\usepackage{algorithm}
\usepackage{algorithmic}
\makeatletter
\renewcommand{\ALG@name}{Algoritmo}
\makeatother

% disponi numeri di pagina
\usepackage{fancyhdr}
\fancyhf{} 
\fancyfoot[L]{\sffamily{\thepage}}

\makeatletter
\fancyhead[L]{\raisebox{1ex}[0pt][0pt]{\sffamily{\@title \ \@date}}} 
\fancyhead[R]{\raisebox{1ex}[0pt][0pt]{\sffamily{\@author}}}
\makeatother

% disegni
\usepackage{pgfplots}
\pgfplotsset{width=10cm,compat=1.9}

\begin{document}
% sezione (data)
\section{Lezione del 23-09-24}

% stili pagina
\thispagestyle{empty}
\pagestyle{fancy}

% testo
\subsection{Introduzione}

\subsubsection{Programma del corso}
Il corso di ricerca operativa si divide in 4 parti:

\begin{enumerate}
	\item Modello di Programmazione Lineare;
	\item Programmazione Lineare su reti, ergo programmazione lineare su grafi;
	\item Programmazione Lineare intera, ergo programmazione lineare col vincolo $x \in \mathbb{Z}^n$;
	\item Programmazione Non Lineare.
\end{enumerate}

Le prime 3 parti hanno come prerequisiti l'algebra lineare: in particolare operazioni matriciali, prodotti scalari, sistemi lineari, teorema di Rouché-Capelli.
La quarta parte richiede invece conoscenze di Analisi II.

\subsubsection{Un problema di programmazione lineare}

La ricerca operativa si occupa di risolvere problemi di ottimizzazione con variabili decisionali e risorse limitate.
Poniamo un problema di esempio:

\begin{problem}{Produzione}
Una ditta produce due prodotti: \textbf{laminato A} e \textbf{laminato B}.
Ogni prodotto deve passare attraverso diversi reparti: il reparto \textbf{materie prime}, il reparto \textbf{taglio}, il reparto \textbf{finiture A} e il reparto \textbf{finiture B}.
Il guadagno è rispettivamente di 8.4 e 11.2 (unità di misura irrilevante) per ogni tipo di laminato.

Ora, nel reparto materie prime, il laminato A occupa 30, ore, e lo B 20 ore.
Nel reparto taglio il laminato A occupa 10 ore e lo B 20 ore.
Il laminato A occupa poi 20 ore nel reparto finiture A, mentre il laminato B occupa 30 ore nel reparto finiture B.
I reparti hanno a disposizione, rispettivamente, 120, 80, 62 e 105 ore.
Possiamo porre queste informazioni in forma tabulare:

	\center \rowcolors{2}{green!10}{green!40!black!20}
	\begin{tabular} { | c || c | c | c | }
		\hline
		\bfseries Reparto & \bfseries Capienza & \bfseries Laminato A & \bfseries Laminato B \\
		\hline 
		Materie prime & 120 & 30 & 20 \\
		Taglio & 80 & 10 & 20 \\
		Finiture A & 62 & 20 & / \\
		Finiture B & 105 & / & 30 \\
		\hline
		\textbf{Guadagno} & & 8.4 & 11.2 \\
		\hline
	\end{tabular}

	\par\bigskip

Quello che ci interessa è chiaramente massimizzare il guadagno.
\end{problem}

Decidiamo di modellizzare questa situazione con un modello matematico.

Il guadagno che abbiamo dai laminati rappresenta una \textbf{funzione obiettivo}, ovvero la funzione che vogliamo ottimizzare.
Ottimizzare significa trovare il modo migliore di massimizzare o minimizzare i valori della funzione agendo sulle variabili decisionali.
La funzione obiettivo va ottimizzata rispettando determinati \textbf{vincoli}, che modellizzano il fatto che le risorse sono limitate.
Una \textbf{soluzione ammissibile} è una qualsiasi soluzione che rispetta i vincoli del problema.
Chiamiamo quindi \textbf{regione ammissibile} l'insieme di tutte le soluzioni ammissibili.
All'interno della regione ammissibile c'è la soluzione che cerchiamo, ovvero la \textbf{soluzione ottima}.

Decidiamo quindi le \textbf{variabili decisionali}, ed esplicitiamo la funzione obiettivo e i vincoli.

In questo caso le variabili decisionali saranno le quantità di laminato A e B da produrre, che individuano un punto in $ \mathbb{R}^2 $ denominato $ ( x_A, x_B ) $. 
Decidere di usare la soluzione $ (1,1) $ significa decidere di produrre 1 unità di laminato A e 1 unità di laminato B, per un guadagno complessivo di $ 8.4 + 11.2 = 19.6 $.

La funzione obiettivo sarà quindi:

$$ f(x_A, x_B) = 8.4 x_A + 11.2 x_B, \quad f: \mathbb{R}^2 \rightarrow \mathbb{R} $$

lineare, e noi saremo interessati a:

$$ \max(f(x_A, x_B)) $$

rispettando i vincoli, ergo nella regione ammissibile.
Per esprimere questi vincoli, cioè il tempo limitato all'interno di ogni reparto, introduciamo il sistema di disequazioni:

\[
	\begin{cases}
		30 x_A + 20 x_B \leq 120 \\
		10 x_A + 20 x_B \leq 80	\\
		20 x_A + 0 x_B \leq 62 \\	
		0 x_A + 30 x_B \leq 105 \\
		- x_A \leq 0 \\
		- x_B \leq 0 \\
	\end{cases}
\]

dove notiamo le ultime due disequazioni indicano la positività di $x_A$ e $x_B$, in forma $ f(x_A, x_B) \leq b $.
Questo sistema non indica altro che la regione ammissibile.

Possiamo riscrivere questo modello usando la notazione dell'algebra lineare.
La funzione obiettiva e i vincoli diventano semplicemente:

\[
	\begin{cases}
		\max(c^T \cdot x) \\
		A \cdot x \leq b	
	\end{cases}
\]

dove $c$ rappresenta il vettore dei costi, $A$ rappresenta la matrice dei costi a $b$ il vettore dei vincoli.
$c$ è trasposto per indicare prodotto fra vettori.

Possiamo scrivere $A$, $b$ e $c$ per esteso:

$$
A:
\begin{pmatrix}
	30 & 20 \\
	10 & 20 \\
	20 & 0 \\
	0 & 30 \\
	-1 & 0 \\
	0 & -1
\end{pmatrix}, \quad
b:
\begin{pmatrix}
	120 \\
	80 \\
	62 \\ 
	105 \\ 
	0 \\ 
	0 
\end{pmatrix}, \quad 
c:
\begin{pmatrix}
	8.4 \\
	11.2 \\
\end{pmatrix}
$$

Notiamo come $A$ e $b$ hanno dimensione verticale $ 4 + 2 = 6 $, dai 4 vincoli superiori e i 2 vincoli inferiori.

A questo punto, possiamo disegnare la regione ammissibile come l'intersezione dei semipiani individuati da ogni singola disuguaglianza.

Si riporta un grafico:

\begin{center}

\begin{tikzpicture}
\begin{axis}[
    axis lines = middle,
    xlabel = {$x_A$},
    ylabel = {$x_B$},
    xmin=0, xmax=7.9,
    ymin=0, ymax=3.9,
    domain=0:10,
    samples=100,
    width=15cm, height=7cm,
    legend pos=north east
  ]

% regione ammissibile

	\addplot[fill=gray, opacity=0.4, forget plot] 
    coordinates {
			(0, 0)
			(3.1, 0)
			(3.1, 1.35)
			(2,3)
			(1, 3.5)
			(0, 3.5)
		};

% rette

\addplot[domain=2:3.1, thick, blue] {6 - 1.5*x}; 
\addlegendentry{$ 30 x_A + 20 x_B \leq 120 $}

\addplot[domain=1:2, thick, green] {4 - 0.5*x}; 
\addlegendentry{$ 10 x_A + 20 x_B \leq 80 $}

\addplot[thick, purple] coordinates {(3.1, 0) (3.1, 1.35)};
\addlegendentry{$ 20 x_A + 0 x_B \leq 62 $}

\addplot[domain=0:1, thick, red] {3.5}; 
\addlegendentry{$ 0 x_A + 30 x_B \leq 105 $}
	
\end{axis}
\end{tikzpicture}

\end{center}

In diversi colori sono riportati i margini delle disequazioni, mentre in grigio è evidenziata la regione ammissibile.
Qualsiasi punto all'interno della regione ammissibile vale come soluzione, e almeno uno di essi è soluzione ottimale.

\par\smallskip

Il modello finora descritto prende il nome di modello di programmazione lineare, e permette di formulare problemi di programmazione lineare (LP).

\begin{definition}{Problema di programmazione lineare (1)}
Un problema di programmazione lineare (LP) riguarda l'ottimizzazione di una funzione lineare in più variabili
soggetta a vincoli di tipo $ =, \ \leq $ e $ \geq $, ovvero in forma:
\[
	\begin{cases}
			\min / \max(c^T \cdot x) \\
			A_i x \leq b \\
			B_j x \geq d \\
			C_k x = e \\
	\end{cases}
\]
\end{definition}

"Programmazione" qui non ha alcun legame col concetto di programmazione informatica, ma si riferisce al fatto che il modello è effettivamente \textit{programmabile}.

"Lineare" si riferisce alla linearità dei problemi che ci permette di risolvere (e quindi del modello).




\end{document}

\documentclass[a4paper,11pt]{article}
\usepackage[a4paper, margin=8em]{geometry}

% usa i pacchetti per la scrittura in italiano
\usepackage[french,italian]{babel}
\usepackage[T1]{fontenc}
\usepackage[utf8]{inputenc}
\frenchspacing 

% usa i pacchetti per la formattazione matematica
\usepackage{amsmath, amssymb, amsthm, amsfonts}

% usa altri pacchetti
\usepackage{gensymb}
\usepackage{hyperref}
\usepackage{standalone}

% imposta il titolo
\title{Appunti Ricerca Operativa}
\author{Luca Seggiani}
\date{23-09-24}

% imposta lo stile
% usa helvetica
\usepackage[scaled]{helvet}
% usa palatino
\usepackage{palatino}
% usa un font monospazio guardabile
\usepackage{lmodern}

\renewcommand{\rmdefault}{ppl}
\renewcommand{\sfdefault}{phv}
\renewcommand{\ttdefault}{lmtt}

% disponi teoremi
\usepackage{tcolorbox}
\newtcolorbox[auto counter, number within=section]{theorem}[2][]{%
	colback=blue!10, 
	colframe=blue!40!black, 
	sharp corners=northwest,
	fonttitle=\sffamily\bfseries, 
	title=Teorema~\thetcbcounter: #2, 
	#1
}

% disponi definizioni
\newtcolorbox[auto counter, number within=section]{definition}[2][]{%
	colback=red!10,
	colframe=red!40!black,
	sharp corners=northwest,
	fonttitle=\sffamily\bfseries,
	title=Definizione~\thetcbcounter: #2,
	#1
}

% disponi problemi
\newtcolorbox[auto counter, number within=section]{problem}[2][]{%
	colback=green!10,
	colframe=green!40!black,
	sharp corners=northwest,
	fonttitle=\sffamily\bfseries,
	title=Problema~\thetcbcounter: #2,
	#1
}

% disponi codice
\usepackage{listings}
\usepackage[table]{xcolor}

\lstdefinestyle{codestyle}{
		backgroundcolor=\color{black!5}, 
		commentstyle=\color{codegreen},
		keywordstyle=\bfseries\color{magenta},
		numberstyle=\sffamily\tiny\color{black!60},
		stringstyle=\color{green!50!black},
		basicstyle=\ttfamily\footnotesize,
		breakatwhitespace=false,         
		breaklines=true,                 
		captionpos=b,                    
		keepspaces=true,                 
		numbers=left,                    
		numbersep=5pt,                  
		showspaces=false,                
		showstringspaces=false,
		showtabs=false,                  
		tabsize=2
}

\lstdefinestyle{shellstyle}{
		backgroundcolor=\color{black!5}, 
		basicstyle=\ttfamily\footnotesize\color{black}, 
		commentstyle=\color{black}, 
		keywordstyle=\color{black},
		numberstyle=\color{black!5},
		stringstyle=\color{black}, 
		showspaces=false,
		showstringspaces=false, 
		showtabs=false, 
		tabsize=2, 
		numbers=none, 
		breaklines=true
}

\lstdefinelanguage{javascript}{
	keywords={typeof, new, true, false, catch, function, return, null, catch, switch, var, if, in, while, do, else, case, break},
	keywordstyle=\color{blue}\bfseries,
	ndkeywords={class, export, boolean, throw, implements, import, this},
	ndkeywordstyle=\color{darkgray}\bfseries,
	identifierstyle=\color{black},
	sensitive=false,
	comment=[l]{//},
	morecomment=[s]{/*}{*/},
	commentstyle=\color{purple}\ttfamily,
	stringstyle=\color{red}\ttfamily,
	morestring=[b]',
	morestring=[b]"
}

% disponi sezioni
\usepackage{titlesec}

\titleformat{\section}
	{\sffamily\Large\bfseries} 
	{\thesection}{1em}{} 
\titleformat{\subsection}
	{\sffamily\large\bfseries}   
	{\thesubsection}{1em}{} 
\titleformat{\subsubsection}
	{\sffamily\normalsize\bfseries} 
	{\thesubsubsection}{1em}{}

% disponi alberi
\usepackage{forest}

\forestset{
	rectstyle/.style={
		for tree={rectangle,draw,font=\large\sffamily}
	},
	roundstyle/.style={
		for tree={circle,draw,font=\large}
	}
}

% disponi algoritmi
\usepackage{algorithm}
\usepackage{algorithmic}
\makeatletter
\renewcommand{\ALG@name}{Algoritmo}
\makeatother

% disponi numeri di pagina
\usepackage{fancyhdr}
\fancyhf{} 
\fancyfoot[L]{\sffamily{\thepage}}

\makeatletter
\fancyhead[L]{\raisebox{1ex}[0pt][0pt]{\sffamily{\@title \ \@date}}} 
\fancyhead[R]{\raisebox{1ex}[0pt][0pt]{\sffamily{\@author}}}
\makeatother

% disegni
\usepackage{pgfplots}
\pgfplotsset{width=10cm,compat=1.9}

\begin{document}
% sezione (data)
\section{Lezione del 23-09-24}

% stili pagina
\thispagestyle{empty}
\pagestyle{fancy}

% testo
\subsection{Introduzione}

\subsubsection{Programma del corso}
Il corso di ricerca operativa si divide in 4 parti:

\begin{enumerate}
	\item Modello di Programmazione Lineare;
	\item Programmazione Lineare su reti, ergo programmazione lineare su grafi;
	\item Programmazione Lineare intera, ergo programmazione lineare col vincolo $x \in \mathbb{Z}^n$;
	\item Programmazione Non Lineare.
\end{enumerate}

Le prime 3 parti hanno come prerequisiti l'algebra lineare: in particolare operazioni matriciali, prodotti scalari, sistemi lineari, teorema di Rouché-Capelli.
La quarta parte richiede invece conoscenze di Analisi II.

\subsubsection{Un problema di programmazione lineare}

La ricerca operativa si occupa di risolvere problemi di ottimizzazione con variabili decisionali e risorse limitate.
Poniamo un problema di esempio:

\begin{problem}{Produzione}
Una ditta produce due prodotti: \textbf{laminato A} e \textbf{laminato B}.
Ogni prodotto deve passare attraverso diversi reparti: il reparto \textbf{materie prime}, il reparto \textbf{taglio}, il reparto \textbf{finiture A} e il reparto \textbf{finiture B}.
Il guadagno è rispettivamente di 8.4 e 11.2 (unità di misura irrilevante) per ogni tipo di laminato.

Ora, nel reparto materie prime, il laminato A occupa 30, ore, e lo B 20 ore.
Nel reparto taglio il laminato A occupa 10 ore e lo B 20 ore.
Il laminato A occupa poi 20 ore nel reparto finiture A, mentre il laminato B occupa 30 ore nel reparto finiture B.
I reparti hanno a disposizione, rispettivamente, 120, 80, 62 e 105 ore.
Possiamo porre queste informazioni in forma tabulare:

	\center \rowcolors{2}{green!10}{green!40!black!20}
	\begin{tabular} { | c || c | c | c | }
		\hline
		\bfseries Reparto & \bfseries Capienza & \bfseries Laminato A & \bfseries Laminato B \\
		\hline 
		Materie prime & 120 & 30 & 20 \\
		Taglio & 80 & 10 & 20 \\
		Finiture A & 62 & 20 & / \\
		Finiture B & 105 & / & 30 \\
		\hline
		\textbf{Guadagno} & & 8.4 & 11.2 \\
		\hline
	\end{tabular}

	\par\bigskip

Quello che ci interessa è chiaramente massimizzare il guadagno.
\end{problem}

Decidiamo di modellizzare questa situazione con un modello matematico.

Il guadagno che abbiamo dai laminati rappresenta una \textbf{funzione obiettivo}, ovvero la funzione che vogliamo ottimizzare.
Ottimizzare significa trovare il modo migliore di massimizzare o minimizzare i valori della funzione agendo sulle variabili decisionali.
La funzione obiettivo va ottimizzata rispettando determinati \textbf{vincoli}, che modellizzano il fatto che le risorse sono limitate.
Una \textbf{soluzione ammissibile} è una qualsiasi soluzione che rispetta i vincoli del problema.
Chiamiamo quindi \textbf{regione ammissibile} l'insieme di tutte le soluzioni ammissibili.
All'interno della regione ammissibile c'è la soluzione che cerchiamo, ovvero la \textbf{soluzione ottima}.

Decidiamo quindi le \textbf{variabili decisionali}, ed esplicitiamo la funzione obiettivo e i vincoli.

In questo caso le variabili decisionali saranno le quantità di laminato A e B da produrre, che individuano un punto in $ \mathbb{R}^2 $ denominato $ ( x_A, x_B ) $. 
Decidere di usare la soluzione $ (1,1) $ significa decidere di produrre 1 unità di laminato A e 1 unità di laminato B, per un guadagno complessivo di $ 8.4 + 11.2 = 19.6 $.

La funzione obiettivo sarà quindi:

$$ f(x_A, x_B) = 8.4 x_A + 11.2 x_B, \quad f: \mathbb{R}^2 \rightarrow \mathbb{R} $$

lineare, e noi saremo interessati a:

$$ \max(f(x_A, x_B)) $$

rispettando i vincoli, ergo nella regione ammissibile.
Per esprimere questi vincoli, cioè il tempo limitato all'interno di ogni reparto, introduciamo il sistema di disequazioni:

\[
	\begin{cases}
		30 x_A + 20 x_B \leq 120 \\
		10 x_A + 20 x_B \leq 80	\\
		20 x_A + 0 x_B \leq 62 \\	
		0 x_A + 30 x_B \leq 105 \\
		- x_A \leq 0 \\
		- x_B \leq 0 \\
	\end{cases}
\]

dove notiamo le ultime due disequazioni indicano la positività di $x_A$ e $x_B$, in forma $ f(x_A, x_B) \leq b $.
Questo sistema non indica altro che la regione ammissibile.

Possiamo riscrivere questo modello usando la notazione dell'algebra lineare.
La funzione obiettiva e i vincoli diventano semplicemente:

\[
	\begin{cases}
		\max(c^T \cdot x) \\
		A \cdot x \leq b	
	\end{cases}
\]

dove $c$ rappresenta il vettore dei costi, $A$ rappresenta la matrice dei costi a $b$ il vettore dei vincoli.
$c$ è trasposto per indicare prodotto fra vettori.

Possiamo scrivere $A$, $b$ e $c$ per esteso:

$$
A:
\begin{pmatrix}
	30 & 20 \\
	10 & 20 \\
	20 & 0 \\
	0 & 30 \\
	-1 & 0 \\
	0 & -1
\end{pmatrix}, \quad
b:
\begin{pmatrix}
	120 \\
	80 \\
	62 \\ 
	105 \\ 
	0 \\ 
	0 
\end{pmatrix}, \quad 
c:
\begin{pmatrix}
	8.4 \\
	11.2 \\
\end{pmatrix}
$$

Notiamo come $A$ e $b$ hanno dimensione verticale $ 4 + 2 = 6 $, dai 4 vincoli superiori e i 2 vincoli inferiori.

A questo punto, possiamo disegnare la regione ammissibile come l'intersezione dei semipiani individuati da ogni singola disuguaglianza.

Si riporta un grafico:

\begin{center}

\begin{tikzpicture}
\begin{axis}[
    axis lines = middle,
    xlabel = {$x_A$},
    ylabel = {$x_B$},
    xmin=0, xmax=7.9,
    ymin=0, ymax=3.9,
    domain=0:10,
    samples=100,
    width=15cm, height=7cm,
    legend pos=north east
  ]

% regione ammissibile

	\addplot[fill=gray, opacity=0.4, forget plot] 
    coordinates {
			(0, 0)
			(3.1, 0)
			(3.1, 1.35)
			(2,3)
			(1, 3.5)
			(0, 3.5)
		};

% rette

\addplot[domain=2:3.1, thick, blue] {6 - 1.5*x}; 
\addlegendentry{$ 30 x_A + 20 x_B \leq 120 $}

\addplot[domain=1:2, thick, green] {4 - 0.5*x}; 
\addlegendentry{$ 10 x_A + 20 x_B \leq 80 $}

\addplot[thick, purple] coordinates {(3.1, 0) (3.1, 1.35)};
\addlegendentry{$ 20 x_A + 0 x_B \leq 62 $}

\addplot[domain=0:1, thick, red] {3.5}; 
\addlegendentry{$ 0 x_A + 30 x_B \leq 105 $}
	
\end{axis}
\end{tikzpicture}

\end{center}

In diversi colori sono riportati i margini delle disequazioni, mentre in grigio è evidenziata la regione ammissibile.
Qualsiasi punto all'interno della regione ammissibile vale come soluzione, e almeno uno di essi è soluzione ottimale.

\par\smallskip

Il modello finora descritto prende il nome di modello di programmazione lineare, e permette di formulare problemi di programmazione lineare (LP).

\begin{definition}{Problema di programmazione lineare (1)}
Un problema di programmazione lineare (LP) riguarda l'ottimizzazione di una funzione lineare in più variabili
soggetta a vincoli di tipo $ =, \ \leq $ e $ \geq $, ovvero in forma:
\[
	\begin{cases}
			\min / \max(c^T \cdot x) \\
			A_i x \leq b \\
			B_j x \geq d \\
			C_k x = e \\
	\end{cases}
\]
\end{definition}

"Programmazione" qui non ha alcun legame col concetto di programmazione informatica, ma si riferisce al fatto che il modello è effettivamente \textit{programmabile}.

"Lineare" si riferisce alla linearità dei problemi che ci permette di risolvere (e quindi del modello).




\end{document}


\documentclass[a4paper,11pt]{article}
\usepackage[a4paper, margin=8em]{geometry}

% usa i pacchetti per la scrittura in italiano
\usepackage[french,italian]{babel}
\usepackage[T1]{fontenc}
\usepackage[utf8]{inputenc}
\frenchspacing 

% usa i pacchetti per la formattazione matematica
\usepackage{amsmath, amssymb, amsthm, amsfonts}

% usa altri pacchetti
\usepackage{gensymb}
\usepackage{hyperref}
\usepackage{standalone}

% cose fluttuanti
\usepackage{float}

% imposta il titolo
\title{Appunti Ricerca Operativa}
\author{Luca Seggiani}
\date{2024}

% disegni
\usepackage{pgfplots}
\pgfplotsset{width=10cm,compat=1.9}

% imposta lo stile
% usa helvetica
\usepackage[scaled]{helvet}
% usa palatino
\usepackage{palatino}
% usa un font monospazio guardabile
\usepackage{lmodern}

\renewcommand{\rmdefault}{ppl}
\renewcommand{\sfdefault}{phv}
\renewcommand{\ttdefault}{lmtt}

% disponi il titolo
\makeatletter
\renewcommand{\maketitle} {
	\begin{center} 
		\begin{minipage}[t]{.8\textwidth}
			\textsf{\huge\bfseries \@title} 
		\end{minipage}%
		\begin{minipage}[t]{.2\textwidth}
			\raggedleft \vspace{-1.65em}
			\textsf{\small \@author} \vfill
			\textsf{\small \@date}
		\end{minipage}
		\par
	\end{center}

	\thispagestyle{empty}
	\pagestyle{fancy}
}
\makeatother

% disponi teoremi
\usepackage{tcolorbox}
\newtcolorbox[auto counter, number within=section]{theorem}[2][]{%
	colback=blue!10, 
	colframe=blue!40!black, 
	sharp corners=northwest,
	fonttitle=\sffamily\bfseries, 
	title=Teorema~\thetcbcounter: #2, 
	#1
}

% disponi definizioni
\newtcolorbox[auto counter, number within=section]{definition}[2][]{%
	colback=red!10,
	colframe=red!40!black,
	sharp corners=northwest,
	fonttitle=\sffamily\bfseries,
	title=Definizione~\thetcbcounter: #2,
	#1
}

% disponi problemi
\newtcolorbox[auto counter, number within=section]{problem}[2][]{%
	colback=green!10,
	colframe=green!40!black,
	sharp corners=northwest,
	fonttitle=\sffamily\bfseries,
	title=Problema~\thetcbcounter: #2,
	#1
}

% disponi codice
\usepackage{listings}
\usepackage[table]{xcolor}

\lstdefinestyle{codestyle}{
		backgroundcolor=\color{black!5}, 
		commentstyle=\color{codegreen},
		keywordstyle=\bfseries\color{magenta},
		numberstyle=\sffamily\tiny\color{black!60},
		stringstyle=\color{green!50!black},
		basicstyle=\ttfamily\footnotesize,
		breakatwhitespace=false,         
		breaklines=true,                 
		captionpos=b,                    
		keepspaces=true,                 
		numbers=left,                    
		numbersep=5pt,                  
		showspaces=false,                
		showstringspaces=false,
		showtabs=false,                  
		tabsize=2
}

\lstdefinestyle{shellstyle}{
		backgroundcolor=\color{black!5}, 
		basicstyle=\ttfamily\footnotesize\color{black}, 
		commentstyle=\color{black}, 
		keywordstyle=\color{black},
		numberstyle=\color{black!5},
		stringstyle=\color{black}, 
		showspaces=false,
		showstringspaces=false, 
		showtabs=false, 
		tabsize=2, 
		numbers=none, 
		breaklines=true
}

\lstdefinelanguage{javascript}{
	keywords={typeof, new, true, false, catch, function, return, null, catch, switch, var, if, in, while, do, else, case, break},
	keywordstyle=\color{blue}\bfseries,
	ndkeywords={class, export, boolean, throw, implements, import, this},
	ndkeywordstyle=\color{darkgray}\bfseries,
	identifierstyle=\color{black},
	sensitive=false,
	comment=[l]{//},
	morecomment=[s]{/*}{*/},
	commentstyle=\color{purple}\ttfamily,
	stringstyle=\color{red}\ttfamily,
	morestring=[b]',
	morestring=[b]"
}

% disponi sezioni
\usepackage{titlesec}

\titleformat{\section}
	{\sffamily\Large\bfseries} 
	{\thesection}{1em}{} 
\titleformat{\subsection}
	{\sffamily\large\bfseries}   
	{\thesubsection}{1em}{} 
\titleformat{\subsubsection}
	{\sffamily\normalsize\bfseries} 
	{\thesubsubsection}{1em}{}

% disponi alberi
\usepackage{forest}

\forestset{
	rectstyle/.style={
		for tree={rectangle,draw,font=\large\sffamily}
	},
	roundstyle/.style={
		for tree={circle,draw,font=\large}
	}
}

% disponi algoritmi
\usepackage{algorithm}
\usepackage{algorithmic}
\makeatletter
\renewcommand{\ALG@name}{Algoritmo}
\makeatother

% disponi numeri di pagina
\usepackage{fancyhdr}
\fancyhf{} 
\fancyfoot[L]{\sffamily{\thepage}}

\makeatletter
\fancyhead[L]{\raisebox{1ex}[0pt][0pt]{\sffamily{\@title \ \@date}}} 
\fancyhead[R]{\raisebox{1ex}[0pt][0pt]{\sffamily{\@author}}}
\makeatother

\begin{document}

% sezione (data)
\section{Lezione del 07-10-24}

% stili pagina
\thispagestyle{empty}
\pagestyle{fancy}

% testo
\subsection{Algoritmo del simplesso primale}
Supponiamo di avere un problema LP in formato primale standard con $n = 8$ vincoli, espresso come:

\[
	\begin{cases}
		\max(c^T \cdot x) \\
		Ax \leq b
	\end{cases}
\]

e con poliedro:

\begin{center}
	\begin{tikzpicture}
	\begin{axis}[
			axis lines = middle,
			xlabel = {$x_A$},
			ylabel = {$x_B$},
			xmin=0, xmax=7.9,
			ymin=0, ymax=3.9,
			samples=100,
			width=13cm, height=7cm,
			legend pos=north east
		]

	\addplot[blue, thick] coordinates {(0,1) (1,0)};  
	\addplot[blue, thick] coordinates {(1,0) (2,0)};  
	\addplot[blue, thick] coordinates {(0,1) (0,2)};  
	\addplot[blue, thick] coordinates {(0,2) (1,3)};  
	\addplot[blue, thick] coordinates {(1,3) (2,3)};  
	\addplot[blue, thick] coordinates {(1,3) (2,3)};  
	\addplot[blue, thick] coordinates {(2,3) (3,2)};  
	\addplot[blue, thick] coordinates {(3,2) (3,1)};  
	\addplot[blue, thick] coordinates {(3,1) (2,0)}; 

	\node at (axis cs:0.7,0.7) [anchor=center] {1};
	\node at (axis cs:1.5,0.3) [anchor=center] {2};
	\node at (axis cs:2.3,0.7) [anchor=center] {3};
	\node at (axis cs:2.7,1.5) [anchor=center] {4};
	\node at (axis cs:2.3,2.3) [anchor=center] {5};
	\node at (axis cs:1.5,2.7) [anchor=center] {6};
	\node at (axis cs:0.7,2.3) [anchor=center] {7};
	\node at (axis cs:0.3,1.5) [anchor=center] {8};
		
	\end{axis}
	\end{tikzpicture}
\end{center}

Scegliamo un vertice di partenza, per adesso ad arbitrio: diciamo $\bar{x} = (0, 1)$ (vedremo in seguito un'algoritmo particolare per ricavare un vertice, che ci permetterà anche di determinare se il poliedro è vuoto o meno).
Ci chiediamo se questo vertice $\bar{x}$ è ottimo.
Visto che è vertice, abbiamo che per una matrice $A_B$ e un vettore $b_B$ di base:
$$
\bar{x} = A_B^{-1} b_B
$$
e che possiamo costruire il complementare duale $\bar{y}$, impostando a zero le variabili fuori base e risolvendo il sistema:
$$
\bar{y} = (cA_B^{-1}, 0)
$$
e applicare il test di ottimalità, cioè vedere se:
$$
cA_B^{-1} \geq 0
$$
ergo $\bar{y} \in D$, quindi il complementare duale esiste e il vertice è ottimo.
Se questa condizione risulta verificata, possiamo fermarci, in quanto abbiamo trovato la soluzione ottimale.

In caso contrario, avremo $\exists k \in B$ tale che $\bar{y}_k < 0$.
Dovremo quindi spostarci verso un'altro vertice, magari \textit{adiacente}, che dal punto di vista delle basi, significa cambiare un solo indice di base, conservando gli altri.
Possiamo formalizzare questa affermazione definendo un \textbf{indice uscente} $h$ ed un \textbf{indice entrante} $k$.
Sostituire un indice di base significa effettuare il cambio di base:
$$
B := B \setminus \{h\} \cup \{k\}
$$

Resta la domanda di \textit{quale} spigolo scegliere: in uno spazio vettoriale $\mathbb{R}^n$, ho a disposizione $n$ spigoli che si staccano dallo stesso vertice.
Ovviamente, vorrei scegliere uno spigolo che accresce la funzione obiettivo, e si può dimostrare che ne esiste almeno uno: altrimenti sarei già all'ottimo.
Inoltre, avendo un metodo per scegliere sempre lo spigolo di crescita maggiore potrei dire 2 cose: l'algoritmo tende all'ottimo (il vertice da cui non si staccano spigoli che accrescono la funzione obiettivo), e termina in un numero finito di passi (prima o poi raggiungerà inevitabilmente un vertice che massimizza la funzione).

Prima però dobbiamo chiarire una questione: scegliere un nuovo spigolo significa trovare 2 indici base, uno da eliminare e uno da inserire.
Si può dire che il primo indice, quello uscente, indica anche la direzione di spostamento: allentando un vincolo ci spostiamo sulla semiretta del prossimo.
Allo stesso tempo, scegliere un indice da rimuovere non basta: dobbiamo scegliere quale introdurre, che geometricamente significa capire \textit{quanto} ci possiamo spostare lungo la semiretta prima di uscire dalla regione di ammissibilità.
Vediamo quindi questi due passaggi in ordine.

\begin{itemize}
	\item \textbf{\textsf{Indice uscente}} \\
Diciamo:
$$
W = \left( -A_B^{-1} \right)
$$
e prendiamo le colonne $W^i$ corrispondenti agli indici di base scelti.

Possiamo allora dire che l'equazione degli spigoli dati dalle disequazioni all'indice $i$ sono:
$$
\bar{x} + \lambda W^i
$$

Mettiamo questa equazione nella funzione costo:
$$
c\left( \bar{x} + \lambda W^i \right) = c \bar{x} + \lambda c W^{i}
$$

Qui abbiamo $c\bar{x}$, che è il valore nel vertice, e un'altro termine scalato da $\lambda$.
Ricordiamo poi che $cA_B^{-1} = \bar{y}_B$, e che $W = \left( -A_B^{-1} \right)$, ergo $c W^{i} = -\bar{y}_B$:
$$
c \bar{x} + \lambda c W^{i} = c \bar{x} - \lambda \bar{y}_B
$$
Vogliamo quindi "allentare" l'indice (e il corrispettivo vertice) che ci dà $\bar{y}_i < 0$, in quanto è quello che restituisce un $c W^{i} > 0$, e quindi un accrescimento della funzione. 
Definiamo allora questo indice:
\begin{definition}{Indice uscente primale}
Chiamiamo indice uscente $h$, da una certa soluzione della base $B$:
$$h := \min\{ i \in B \ \text{t.c.} \ \bar{y_i} < 0  \}$$
\end{definition}
Il $\min$ significa che in caso di più $i$ negativi, si adotta la regola anticiclo (di Bland) di scegliere il primo.
In caso di nessun $i$ negativo, la complementare duale esiste e siamo sull'ottimo.

	\item \textbf{\textsf{Indice entrante}} \\
Adesso cerchiamo per quali $\lambda$ lo spigolo $\bar{x} + \lambda W^h$ resta ammissibile, ergo soddisfa:
$$
A_i \left( \bar{x} + \lambda W^h \right) \leq b_i, \quad i \in N 
$$
Questo significa effettivamente vedere qual'è il primo vincolo che "stringiamo", o che incontriamo, spostandoci lungo la semiretta ottenuta allentando il vincolo dato dall'indice uscente.

Possiamo dire:
$$
A_i \left( \bar{x} + \lambda W^h \right) = A_i \bar{x} + \lambda A_i W^h \leq b_i
$$
da cui si ricava (e si risolve) la disequazione di primo grado:
$$
\lambda A_i W^h \leq b_i - A_i \bar{x} \Rightarrow \lambda \leq \frac{b_i - A_i \bar{x}}{A_i W^h}
$$
Notiamo che se fosse $A_i W^h \leq 0, \ \forall i \in N$, avremmo che l'indice rappresenta una direzione di regressione, in quanto $\lambda \rightarrow +\infty$.
Si ha quindi che il duale non ha soluzione, e il primale $\rightarrow +\infty$.
In caso contrario, noi vogliamo trovare il primo vincolo che si va a stringere, quindi dovremo calcolare tutti gli $r_i$:
$$
r_i = \frac{b_i - A_i \bar{x}}{A_i W^h}, \quad i \in N, \quad A_i W^h > 0
$$
e scegliere l'indice che dà $\vartheta = \min(r_i)$.
Definiamo allora anche questo indice:
\begin{definition}{Indice entrante primale}
	Chiamiamo indice entrante $k$, da una certa soluzione della base $B$ e un certo indice uscente $h$:
	$$
	k := \min\{ i \in N \ \text{t.c.} \ A_i W^h > 0, \quad \frac{b_i - A_i \bar{x}}{A_i W^h} = \vartheta \}	
	$$
\end{definition}
Anche qui, il $\min$ serve a selezionare il primo indice valido, ed è una regola anticiclo (di Bland).
Notiamo due possibili situazioni:
\begin{itemize}
	\item Si potrebbero avere più $r_i$ uguali: questi rappresentano soluzioni di base degenere \textit{in arrivo}, in quanto sono più modi di arrivare allo stesso vertice stringendo vincoli diversi;
	\item Si potrebbe avere un $r_i$ nullo: questo significa che il vertice è sullo stesso vertice da dove siamo partiti, ergo rappresenta una soluzione di base degenere \textit{in partenza}.
\end{itemize}
Come prima, le regole anticiclo di Bland assicurano anche che l'algoritmo non si blocchi a ciclare su queste soluzioni degeneri.
\end{itemize}

Abbiamo quindi tutti gli strumenti necessari alla formulazione dell'algoritmo del simplesso:
\begin{algorithm}[H]
\caption{del simplesso primale}
\begin{algorithmic}
	\STATE \textbf{Input:} un problema LP in forma primale standard
	\STATE \textbf{Output:} la soluzione ottima 
	\STATE Trova una base B che genera una soluzione di base primale ammissibile.
	\STATE \textsf{ciclo:}
	\STATE Calcola la soluzione di base primale $\bar{x} = A_b^{-1} b_B$ e la soluzione di base duale $\bar{y} = (cA_b^{-1}, 0)$
	\IF{$\bar{y_B} \geq 0$}
		\STATE Fermati, $\bar{x}$ è ottima per $P$ e $\bar{y}$ è ottima per $D$
k\ELSE
		\STATE Calcola l'indice uscente: 
		$$
		h := \min\{ i \in B \ \text{t.c.} \ \bar{y_i} < 0 \}
		$$
		poni $W := -A_B^{-1}$ e indica con $W^h$ la $h$-esima colonna di $W$
	\ENDIF
	\IF{$A_i W^h \leq 0 \quad \forall i \in N$}
		\STATE Fermati, $P \rightarrow +\infty$ e $D$ non ha soluzione ottima.
	\ELSE
		\STATE Calcola:
		$$
		\vartheta = \min\{ \frac{b_i - A_i \bar{x}}{A_i W^h} \text{t.c.} \quad i \in N, \quad A_i W^h > 0 \}
		$$
		e trova l'indice entrante: 
		$$ 
		h := \min\{ i \in N \ \text{t.c.} \ A_i W^h > 0, \quad \frac{b_i - A_i \bar{x}}{A_i W^h} = \vartheta \} 
		$$
	\ENDIF
	\STATE Aggiorna la base come:
	$$
	B := B \setminus \{h\} \cup \{k\}
	$$
	\STATE Torna a \textsf{ciclo}
\end{algorithmic}
\end{algorithm}

\end{document}


\documentclass[a4paper,11pt]{article}
\usepackage[a4paper, margin=8em]{geometry}

% usa i pacchetti per la scrittura in italiano
\usepackage[french,italian]{babel}
\usepackage[T1]{fontenc}
\usepackage[utf8]{inputenc}
\frenchspacing 

% usa i pacchetti per la formattazione matematica
\usepackage{amsmath, amssymb, amsthm, amsfonts}

% usa altri pacchetti
\usepackage{gensymb}
\usepackage{hyperref}
\usepackage{standalone}

% imposta il titolo
\title{Appunti Ricerca Operativa}
\author{Luca Seggiani}
\date{2024}

% disegni
\usepackage{pgfplots}
\pgfplotsset{width=10cm,compat=1.9}

% imposta lo stile
% usa helvetica
\usepackage[scaled]{helvet}
% usa palatino
\usepackage{palatino}
% usa un font monospazio guardabile
\usepackage{lmodern}

\renewcommand{\rmdefault}{ppl}
\renewcommand{\sfdefault}{phv}
\renewcommand{\ttdefault}{lmtt}

% disponi il titolo
\makeatletter
\renewcommand{\maketitle} {
	\begin{center} 
		\begin{minipage}[t]{.8\textwidth}
			\textsf{\huge\bfseries \@title} 
		\end{minipage}%
		\begin{minipage}[t]{.2\textwidth}
			\raggedleft \vspace{-1.65em}
			\textsf{\small \@author} \vfill
			\textsf{\small \@date}
		\end{minipage}
		\par
	\end{center}

	\thispagestyle{empty}
	\pagestyle{fancy}
}
\makeatother

% disponi teoremi
\usepackage{tcolorbox}
\newtcolorbox[auto counter, number within=section]{theorem}[2][]{%
	colback=blue!10, 
	colframe=blue!40!black, 
	sharp corners=northwest,
	fonttitle=\sffamily\bfseries, 
	title=Teorema~\thetcbcounter: #2, 
	#1
}

% disponi definizioni
\newtcolorbox[auto counter, number within=section]{definition}[2][]{%
	colback=red!10,
	colframe=red!40!black,
	sharp corners=northwest,
	fonttitle=\sffamily\bfseries,
	title=Definizione~\thetcbcounter: #2,
	#1
}

% disponi problemi
\newtcolorbox[auto counter, number within=section]{problem}[2][]{%
	colback=green!10,
	colframe=green!40!black,
	sharp corners=northwest,
	fonttitle=\sffamily\bfseries,
	title=Problema~\thetcbcounter: #2,
	#1
}

% disponi codice
\usepackage{listings}
\usepackage[table]{xcolor}

\lstdefinestyle{codestyle}{
		backgroundcolor=\color{black!5}, 
		commentstyle=\color{codegreen},
		keywordstyle=\bfseries\color{magenta},
		numberstyle=\sffamily\tiny\color{black!60},
		stringstyle=\color{green!50!black},
		basicstyle=\ttfamily\footnotesize,
		breakatwhitespace=false,         
		breaklines=true,                 
		captionpos=b,                    
		keepspaces=true,                 
		numbers=left,                    
		numbersep=5pt,                  
		showspaces=false,                
		showstringspaces=false,
		showtabs=false,                  
		tabsize=2
}

\lstdefinestyle{shellstyle}{
		backgroundcolor=\color{black!5}, 
		basicstyle=\ttfamily\footnotesize\color{black}, 
		commentstyle=\color{black}, 
		keywordstyle=\color{black},
		numberstyle=\color{black!5},
		stringstyle=\color{black}, 
		showspaces=false,
		showstringspaces=false, 
		showtabs=false, 
		tabsize=2, 
		numbers=none, 
		breaklines=true
}

\lstdefinelanguage{javascript}{
	keywords={typeof, new, true, false, catch, function, return, null, catch, switch, var, if, in, while, do, else, case, break},
	keywordstyle=\color{blue}\bfseries,
	ndkeywords={class, export, boolean, throw, implements, import, this},
	ndkeywordstyle=\color{darkgray}\bfseries,
	identifierstyle=\color{black},
	sensitive=false,
	comment=[l]{//},
	morecomment=[s]{/*}{*/},
	commentstyle=\color{purple}\ttfamily,
	stringstyle=\color{red}\ttfamily,
	morestring=[b]',
	morestring=[b]"
}

% disponi sezioni
\usepackage{titlesec}

\titleformat{\section}
	{\sffamily\Large\bfseries} 
	{\thesection}{1em}{} 
\titleformat{\subsection}
	{\sffamily\large\bfseries}   
	{\thesubsection}{1em}{} 
\titleformat{\subsubsection}
	{\sffamily\normalsize\bfseries} 
	{\thesubsubsection}{1em}{}

% disponi alberi
\usepackage{forest}

\forestset{
	rectstyle/.style={
		for tree={rectangle,draw,font=\large\sffamily}
	},
	roundstyle/.style={
		for tree={circle,draw,font=\large}
	}
}

% disponi algoritmi
\usepackage{algorithm}
\usepackage{algorithmic}
\makeatletter
\renewcommand{\ALG@name}{Algoritmo}
\makeatother

% disponi numeri di pagina
\usepackage{fancyhdr}
\fancyhf{} 
\fancyfoot[L]{\sffamily{\thepage}}

\makeatletter
\fancyhead[L]{\raisebox{1ex}[0pt][0pt]{\sffamily{\@title \ \@date}}} 
\fancyhead[R]{\raisebox{1ex}[0pt][0pt]{\sffamily{\@author}}}
\makeatother

\begin{document}

% sezione (data)
\section{Lezione del 08-10-24}

% stili pagina
\thispagestyle{empty}
\pagestyle{fancy}

% testo
Abbiamo trovato un'algoritmo per ricavare una soluzione ottimale partendo da una soluzione di base primale ammissibile.
Nel fare ciò, si è dato per scontato che l'algoritmo di partenza ci avrebbe fornito una soluzione di base primale ammissibile, e che qualsiasi successivo passo del simplesso ci avrebbe restituito altre soluzioni di base primali ammissibili.
In verita, le soluzioni di base trovate possono avere le seguenti combinazioni di ammissibilità su $P$ primale e $D$ duale:
\begin{itemize}
	\item \textbf{Ammissibile su $P$ e $D$:} in questo caso siamo all'ottimo dalla dualità forte;
	\item \textbf{Ammissibile su $P$ ma non su $D$:} in questo caso siamo su un comune punto ammissibile non ottimo;
	\item \textbf{Non ammissibile su $P$ o $D$:} in questo caso si scarta la soluzione;
	\item \textbf{Ammissibile su $D$ ma non su $P$:} effettivamente, ancora non abbiamo un modo per gestire questa situazione.
\end{itemize}

Per questo motivo estendiamo l'algoritmo del simplesso al duale.

\subsection{Algoritmo del simplesso duale}
Intendiamo mantenere, ad ogni passo, la soluzione di base duale come ammissibile e controllare l'ammissibilità di quella primale.
Se la soluzione di base primale è ammissibile (l'inverso di come avevamo visto per il simplesso primale), allora siamo sull'ottimo.
Altrimenti, si cambia base, cercando di minimizzare la funzione obiettivo su una nuova soluzione di base duale.

Abbiamo quindi un vertice del poliedro duale, cioè una soluzione di base ammissibile del duale. 
Ci chiediamo se questo vertice $\bar{y}$ è ottimo.
Visto che è vertice del duale, abbiamo che per una matrice $A_B$ e un vettore costo $c$ del primale:
$$
\bar{y} = (cA_B^{-1}, 0), \quad \text{con} \ cA_B^{-1} \geq 0 
$$
dove ricordiamo questa notazione significa impostare le variabili non di base $\in N$ a $0$ e risolvere il sistema in $m$ variabili rimasto.

Adesso possiamo costruire il complementare primale $\bar{x}$, ponendo:
$$
\bar{x} = A_B^{-1}b_b 
$$
e applicare il test di ammissibilità, cioè vedere se:
$$
A_N(A_B^{-1} b_B) \leq b_N
$$
ergo $\bar{x} \in P$, quindi il complementare duale esiste e il vertice è ottimo.
Come prima, se questa condizione risulta verificata possiamo fermarci, in quanto abbiamo trovato la soluzione ottimale.

In caso contrario, avremo $\exists i \in N$ tale che $b_i - A_i (A_B^{-1} - b_B) < 0$.
Questo equivale a ciò che avevamo trovato per il primale per quanto riguardava gli indici uscenti, con una sola differenza: per risolvere il duale, si trova \textbf{prima} l'\textbf{indice entrante}, e \textbf{poi} l'\textbf{indice uscente}.
Vediamo quindi i passaggi in ordine:

\begin{itemize}
	\item \textbf{\textsf{Indice entrante}} \\
Abbiamo che, se il vertice non è ottimo, vale:
$$
\exists i \in N \ \text{t.c.} \ b_i - A_i (A_B^{-1} - b_B) < 0
$$
Vogliamo che l'indice entrante sia quell che raggiunge questo valore negativo, quindi:
\begin{definition}{Indice entrante duale}
Chiamiamo indice entrante $k$, da una certa soluzione della base $B$:
$$k := \min\{ i \in N \ \text{t.c.} \ b_i - A_i \bar{x} < 0 \}$$
\end{definition}
Come sempre, $\min$ significa che in caso di più $i$ negativi, si adotta la regola anticiclo (di Bland) di scegliere il primo.
In caso di nessun $i$ negativo, la complementare duale esiste e siamo sull'ottimo.

	\item \textbf{\textsf{Indice uscente}} \\
Cerchiamo quindi l'indice uscente.
Dovremo prima definire la matrice $W$ come $ W = -A_B^{-1}$, e  calcolare il prodotto (perlopiù analogo al primale, ma si noti il segno della diseguaglianza capovolto):
$$ 
A_k W^i < 0
$$

Questo ci fornisce una regola, come nel simplesso primale, per l'esistenza di una soluzione: nel caso non sia verificato, si ha che il duale $\rightarrow -\infty$ e che il primale è vuoto.
In caso contrario, si possono calcolare i rapporti, come:
$$
r_i = \frac{-\bar{y}_i}{A_kW^i}, \quad i \in B, A_k W^i < 0
$$
e scegliere l'indice che da $\vartheta = \min{r_i}$, cioè:
\begin{definition}{Indice uscente duale}
	Chiamiamo indice uscente $h$, da una certa soluzione della base $B$ e un certo indice entrante $k$:
	$$
	h := \min\{ i \in B \ \text{t.c.} \ A_k W^i < 0, \quad \frac{-\bar{y}_i}{A_kW^i} = \vartheta \}	
	$$
\end{definition}
Anche qui, il $\min$ serve a selezionare il primo indice valido, ed è una regola anticiclo.
\end{itemize}

Abbiamo quindi revisionato tutti gli strumenti necessari alla formulazione dell'algoritmo del simplesso, stavolta duale:
\begin{algorithm}[H]
\caption{del simplesso duale}
\begin{algorithmic}
	\STATE \textbf{Input:} un problema LP in forma primale standard
	\STATE \textbf{Output:} la soluzione ottima 
	\STATE Trova una base B che genera una soluzione di base duale ammissibile.
	\STATE \textsf{ciclo:}
	\STATE Calcola la soluzione di base duale $\bar{y} = (cA_B^{-1}, 0)$ e la soluzione di base primale $\bar{x} = A_B^{-1}b_B$
	\IF{$A_N(A_B^{-1} b_B) \leq b_N$}
		\STATE Fermati, $\bar{y}$ è ottima per $D$ e $\bar{x}$ è ottima per $P$
	\ELSE
		\STATE Calcola l'indice entrante: 
		$$
		k := \min\{ i \in N \ \text{t.c.} \ b_i - A_i \bar{x} < 0 \}
		$$
		poni $W := -A_B^{-1}$ e indica con $W^i$ la $i$-esima colonna di $W$
	\ENDIF
	\IF{$A_k W^i \geq 0 \quad \forall i \in B$}
		\STATE Fermati, $D \rightarrow -\infty$ e $P$ non ha soluzione ottima.
	\ELSE
		\STATE Calcola:
		$$
		\vartheta = \min\{ \frac{-\bar{y}_i}{A_kW^i} \text{t.c.} \quad i \in B, \quad A_k W^i < 0 \}
		$$
		e trova l'indice uscente: 
		$$ 
		h := \min\{ i \in B \ \text{t.c.} \ A_k W^i < 0, \quad \frac{-\bar{y}_i}{A_kW^i} = \vartheta \}	
		$$
	\ENDIF
	\STATE Aggiorna la base come:
	$$
	B := B \setminus \{h\} \cup \{k\}
	$$
	\STATE Torna a \textsf{ciclo}
\end{algorithmic}
\end{algorithm}

\end{document}



\documentclass[a4paper,11pt]{article}
\usepackage[a4paper, margin=8em]{geometry}

% usa i pacchetti per la scrittura in italiano
\usepackage[french,italian]{babel}
\usepackage[T1]{fontenc}
\usepackage[utf8]{inputenc}
\frenchspacing 

% usa i pacchetti per la formattazione matematica
\usepackage{amsmath, amssymb, amsthm, amsfonts}

% usa altri pacchetti
\usepackage{gensymb}
\usepackage{hyperref}
\usepackage{standalone}

% imposta il titolo
\title{Appunti Ricerca Operativa}
\author{Luca Seggiani}
\date{2024}

% disegni
\usepackage{pgfplots}
\pgfplotsset{width=10cm,compat=1.9}

% imposta lo stile
% usa helvetica
\usepackage[scaled]{helvet}
% usa palatino
\usepackage{palatino}
% usa un font monospazio guardabile
\usepackage{lmodern}

\renewcommand{\rmdefault}{ppl}
\renewcommand{\sfdefault}{phv}
\renewcommand{\ttdefault}{lmtt}

% disponi il titolo
\makeatletter
\renewcommand{\maketitle} {
	\begin{center} 
		\begin{minipage}[t]{.8\textwidth}
			\textsf{\huge\bfseries \@title} 
		\end{minipage}%
		\begin{minipage}[t]{.2\textwidth}
			\raggedleft \vspace{-1.65em}
			\textsf{\small \@author} \vfill
			\textsf{\small \@date}
		\end{minipage}
		\par
	\end{center}

	\thispagestyle{empty}
	\pagestyle{fancy}
}
\makeatother

% disponi teoremi
\usepackage{tcolorbox}
\newtcolorbox[auto counter, number within=section]{theorem}[2][]{%
	colback=blue!10, 
	colframe=blue!40!black, 
	sharp corners=northwest,
	fonttitle=\sffamily\bfseries, 
	title=Teorema~\thetcbcounter: #2, 
	#1
}

% disponi definizioni
\newtcolorbox[auto counter, number within=section]{definition}[2][]{%
	colback=red!10,
	colframe=red!40!black,
	sharp corners=northwest,
	fonttitle=\sffamily\bfseries,
	title=Definizione~\thetcbcounter: #2,
	#1
}

% disponi problemi
\newtcolorbox[auto counter, number within=section]{problem}[2][]{%
	colback=green!10,
	colframe=green!40!black,
	sharp corners=northwest,
	fonttitle=\sffamily\bfseries,
	title=Problema~\thetcbcounter: #2,
	#1
}

% disponi codice
\usepackage{listings}
\usepackage[table]{xcolor}

\lstdefinestyle{codestyle}{
		backgroundcolor=\color{black!5}, 
		commentstyle=\color{codegreen},
		keywordstyle=\bfseries\color{magenta},
		numberstyle=\sffamily\tiny\color{black!60},
		stringstyle=\color{green!50!black},
		basicstyle=\ttfamily\footnotesize,
		breakatwhitespace=false,         
		breaklines=true,                 
		captionpos=b,                    
		keepspaces=true,                 
		numbers=left,                    
		numbersep=5pt,                  
		showspaces=false,                
		showstringspaces=false,
		showtabs=false,                  
		tabsize=2
}

\lstdefinestyle{shellstyle}{
		backgroundcolor=\color{black!5}, 
		basicstyle=\ttfamily\footnotesize\color{black}, 
		commentstyle=\color{black}, 
		keywordstyle=\color{black},
		numberstyle=\color{black!5},
		stringstyle=\color{black}, 
		showspaces=false,
		showstringspaces=false, 
		showtabs=false, 
		tabsize=2, 
		numbers=none, 
		breaklines=true
}

\lstdefinelanguage{javascript}{
	keywords={typeof, new, true, false, catch, function, return, null, catch, switch, var, if, in, while, do, else, case, break},
	keywordstyle=\color{blue}\bfseries,
	ndkeywords={class, export, boolean, throw, implements, import, this},
	ndkeywordstyle=\color{darkgray}\bfseries,
	identifierstyle=\color{black},
	sensitive=false,
	comment=[l]{//},
	morecomment=[s]{/*}{*/},
	commentstyle=\color{purple}\ttfamily,
	stringstyle=\color{red}\ttfamily,
	morestring=[b]',
	morestring=[b]"
}

% disponi sezioni
\usepackage{titlesec}

\titleformat{\section}
	{\sffamily\Large\bfseries} 
	{\thesection}{1em}{} 
\titleformat{\subsection}
	{\sffamily\large\bfseries}   
	{\thesubsection}{1em}{} 
\titleformat{\subsubsection}
	{\sffamily\normalsize\bfseries} 
	{\thesubsubsection}{1em}{}

% disponi alberi
\usepackage{forest}

\forestset{
	rectstyle/.style={
		for tree={rectangle,draw,font=\large\sffamily}
	},
	roundstyle/.style={
		for tree={circle,draw,font=\large}
	}
}

% disponi algoritmi
\usepackage{algorithm}
\usepackage{algorithmic}
\makeatletter
\renewcommand{\ALG@name}{Algoritmo}
\makeatother

% disponi numeri di pagina
\usepackage{fancyhdr}
\fancyhf{} 
\fancyfoot[L]{\sffamily{\thepage}}

\makeatletter
\fancyhead[L]{\raisebox{1ex}[0pt][0pt]{\sffamily{\@title \ \@date}}} 
\fancyhead[R]{\raisebox{1ex}[0pt][0pt]{\sffamily{\@author}}}
\makeatother

\begin{document}

% sezione (data)
\section{Lezione del 09-10-24}

% stili pagina
\thispagestyle{empty}
\pagestyle{fancy}

% testo
\subsection{Poliedro vuoto}
Fino ad ora abbiamo rimandato la trattazione dell'algoritmo per determinare se un poliedro è vuoto.
Diamo adesso quest'algoritmo, notando inoltre che, nel caso il poliedro non fosse vuoto, dovrebbe fornirci un possibile vertice di partenza per l'algoritmo del simplesso.

Partiamo da un poliedro in forma duale standard:
\[
	\begin{cases}
		Ax = b \\
		x \geq 0
	\end{cases}
\]
dove $m$ è il numero di variabili e $n$ il numero di vincoli, e quindi $A: m \times n$.

Adottiamo questa forma in quanto siamo sicuri (a differenza della forma primale) che non presenti linealità.
Costruiamo quindi quello che viene chiamato \textbf{duale ausiliario}, $\text{DA}$:
\[
	\begin{cases}	
		Ax = b \\
		x \geq 0
	\end{cases}
	\rightarrow
	\begin{cases}
		\min \sum_{i=1}^n	\varepsilon_i \\ 
		Ax + I\varepsilon = b \\
		x_i \varepsilon \geq 0
	\end{cases}
\]
dove $\varepsilon$ rappresenta un vettore di variabili ausiliarie di dimensione $n$, cioè una per ogni equazione.
Possiamo rappresentare il sistema ottenuto anche attraverso due matrici a blocchi:
$$
\left( A | I \right) \left( x \over \varepsilon \right) 
$$

Chiamiamo quindi $v(\text{DA})$ la soluzione del duale ausiliario, e formuliamo il teorema:
\begin{theorem}{}
	Se $v(\text{DA}) > 0$, allora $D = \emptyset$, altrimenti, se $v(\text{DA}) = 0$, $D \neq \emptyset$.
	Equivale a dire:
	$$
	v(\text{DA}) = 0 \Leftrightarrow D \neq \emptyset
	$$
\end{theorem}

Dobbiamo però capire come risolvere il duale ausiliario, magari senza usare questo teorema, in quanto questo si andrebbe a creare una catena infinita di duali ausiliari da risolvere...
Notiamo quindi che il duale ausiliario ha sempre una base plausibile, che è quella data dagli indici delle ultime $n$ variabili, quelle introdotte come $\varepsilon_i$.

Non solo, abbiamo anche che:
\begin{theorem}{}
	La soluzione ottima di $\text(DA)$, se $D \neq \emptyset$, ci fornisce un vertice di $D$ stesso.
\end{theorem}
Notiamo che, visto che il primo teorema chiedeva $v(\text{DA}) = 0$ perché $D \neq \emptyset$, allora si ha che nella soluzione ottima di $\text{DA}$ nulla (quella che dimostra la non vuotezza del poliedro), le variabili $\varepsilon_i$ sono nulle.

Il procedimento sarà quindi:
\begin{enumerate}
	\item Prendere il problema duale;
	\item Ricavare il duale ausiliario inserendo nei vincoli il vettore di $n$ variabili ausiliarie $\varepsilon$;
	\item Risolvere il duale ausiliario attraverso il simplesso duale, prendendo come passo iniziale la base data dagli ultimi $n$ indici, cioè che comprende le variabili ausiliarie appena introdotte;
	\item Fare $\geq n$ passi al simplesso, aspettandoci che i primi $n$ passi rimuovano tutte le variabili ausiliarie.
\end{enumerate}

Il vertice che ricaviamo dall'algoritmo prende il nome di \textbf{soluzione ammissibile di base}.

\subsection{Ricavare le variazioni dai cambi di base}
Abbiamo che svolgere un passo al simplesso (primale o duale) significa effettuare un cambio di base, con base entrante e base uscente.
Possiamo calcolare valori definiti, che la soluzione di base sia ammissibile o meno, per la funzione obiettivo del primale e del duale per qualsiasi base.

Potremmo voler calcolare qual'è la variazione di valore di questa funzione dato un certo cambio di base.
Ricordiamo quindi di aver introdotto la formula per spostarci lungo i vincoli di un certo $\lambda$ sul primale:
$$
cx(\lambda) = c\bar{x} - \lambda\bar{y}_h
$$
e sul duale:
$$
by(\lambda) = b\bar{y} + \lambda\left( b_h - A_k \bar{x} \right)
$$

Queste due formule stabiliscono un legame fra il valore della funzione obiettivo e quello di determinate soluzioni $\bar{x}$ e $\bar{y}$ sottoposte a perturbazioni nella direzione dei vertici entranti (cioè allentando il vincolo $h$) di valore $\lambda$.
Possiamo quindi usarle per calcolare quanto di chiedevamo all'inizio del paragrafo: la variazione del valore delle funzioni obiettivo del primale e del duale dopo il cambio di base, ricordando che il nostro $\lambda$ sarà il rapporto $r_i$ (in particolare lo avevamo chiamato $\vartheta$) scelto per determinare il vincolo entrante (primale) o uscente (duale):
$$
r_p = \frac{b_k - A_k \bar{x}}{A_k W^h}, \quad r_d = \frac{-\bar{y}_h}{A_k W^h}
$$

\subsection{Casi degeneri}
Concludiamo infine la trattazione della PL notando il motivo dell'uso delle regole anticiclo di Bland nel calcolo degli indici uscenti ed entranti nell'algoritmo del simplesso.
Prendiamo un problema di esempio:

\[
	\begin{cases}
		\max(x_1 + x_2) \\ 
		x_1 \leq 1 \\ 
		x_2 \leq 1 \\ 
		x_1 + x_2 \leq 2 \\ 
		x_i \geq 0
	\end{cases}
\]

Si ha che il vertice del poliedro $\bar{x} = (1,1)$ è degenere: possiamo ottenerlo dalle basi $B = \{ 1, 2 \}$, $B = \{ 1, 3 \}$ e $B = \{ 2, 3\}$.
In questo caso tutto funziona perché il vertice è anche ottimo, in caso contrario potremmo, se non si usassero le regole anticiclo di Bland, finire in un ciclo infinito.

Regole alternative a quella di Bland sono la scelta dell'indice sempre maggiore, o degli scarti o rapporti (apparentemente) ottimi, cioè più piccoli:
$$
b_k - A_k \bar{x} = \min_{i \in N}\left( b_i - A_i \bar{x} \right)
$$
Queste regole, sopratutto l'ultima, potrebbero sembrare equivalenti o migliori di quelle di Bland.
Invece è importante ricordare che potrebbero causare cicli.

\end{document}

\end{document}