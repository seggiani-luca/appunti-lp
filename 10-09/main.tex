
\documentclass[a4paper,11pt]{article}
\usepackage[a4paper, margin=8em]{geometry}

% usa i pacchetti per la scrittura in italiano
\usepackage[french,italian]{babel}
\usepackage[T1]{fontenc}
\usepackage[utf8]{inputenc}
\frenchspacing 

% usa i pacchetti per la formattazione matematica
\usepackage{amsmath, amssymb, amsthm, amsfonts}

% usa altri pacchetti
\usepackage{gensymb}
\usepackage{hyperref}
\usepackage{standalone}

% imposta il titolo
\title{Appunti Ricerca Operativa}
\author{Luca Seggiani}
\date{2024}

% disegni
\usepackage{pgfplots}
\pgfplotsset{width=10cm,compat=1.9}

% imposta lo stile
% usa helvetica
\usepackage[scaled]{helvet}
% usa palatino
\usepackage{palatino}
% usa un font monospazio guardabile
\usepackage{lmodern}

\renewcommand{\rmdefault}{ppl}
\renewcommand{\sfdefault}{phv}
\renewcommand{\ttdefault}{lmtt}

% disponi il titolo
\makeatletter
\renewcommand{\maketitle} {
	\begin{center} 
		\begin{minipage}[t]{.8\textwidth}
			\textsf{\huge\bfseries \@title} 
		\end{minipage}%
		\begin{minipage}[t]{.2\textwidth}
			\raggedleft \vspace{-1.65em}
			\textsf{\small \@author} \vfill
			\textsf{\small \@date}
		\end{minipage}
		\par
	\end{center}

	\thispagestyle{empty}
	\pagestyle{fancy}
}
\makeatother

% disponi teoremi
\usepackage{tcolorbox}
\newtcolorbox[auto counter, number within=section]{theorem}[2][]{%
	colback=blue!10, 
	colframe=blue!40!black, 
	sharp corners=northwest,
	fonttitle=\sffamily\bfseries, 
	title=Teorema~\thetcbcounter: #2, 
	#1
}

% disponi definizioni
\newtcolorbox[auto counter, number within=section]{definition}[2][]{%
	colback=red!10,
	colframe=red!40!black,
	sharp corners=northwest,
	fonttitle=\sffamily\bfseries,
	title=Definizione~\thetcbcounter: #2,
	#1
}

% disponi problemi
\newtcolorbox[auto counter, number within=section]{problem}[2][]{%
	colback=green!10,
	colframe=green!40!black,
	sharp corners=northwest,
	fonttitle=\sffamily\bfseries,
	title=Problema~\thetcbcounter: #2,
	#1
}

% disponi codice
\usepackage{listings}
\usepackage[table]{xcolor}

\lstdefinestyle{codestyle}{
		backgroundcolor=\color{black!5}, 
		commentstyle=\color{codegreen},
		keywordstyle=\bfseries\color{magenta},
		numberstyle=\sffamily\tiny\color{black!60},
		stringstyle=\color{green!50!black},
		basicstyle=\ttfamily\footnotesize,
		breakatwhitespace=false,         
		breaklines=true,                 
		captionpos=b,                    
		keepspaces=true,                 
		numbers=left,                    
		numbersep=5pt,                  
		showspaces=false,                
		showstringspaces=false,
		showtabs=false,                  
		tabsize=2
}

\lstdefinestyle{shellstyle}{
		backgroundcolor=\color{black!5}, 
		basicstyle=\ttfamily\footnotesize\color{black}, 
		commentstyle=\color{black}, 
		keywordstyle=\color{black},
		numberstyle=\color{black!5},
		stringstyle=\color{black}, 
		showspaces=false,
		showstringspaces=false, 
		showtabs=false, 
		tabsize=2, 
		numbers=none, 
		breaklines=true
}

\lstdefinelanguage{javascript}{
	keywords={typeof, new, true, false, catch, function, return, null, catch, switch, var, if, in, while, do, else, case, break},
	keywordstyle=\color{blue}\bfseries,
	ndkeywords={class, export, boolean, throw, implements, import, this},
	ndkeywordstyle=\color{darkgray}\bfseries,
	identifierstyle=\color{black},
	sensitive=false,
	comment=[l]{//},
	morecomment=[s]{/*}{*/},
	commentstyle=\color{purple}\ttfamily,
	stringstyle=\color{red}\ttfamily,
	morestring=[b]',
	morestring=[b]"
}

% disponi sezioni
\usepackage{titlesec}

\titleformat{\section}
	{\sffamily\Large\bfseries} 
	{\thesection}{1em}{} 
\titleformat{\subsection}
	{\sffamily\large\bfseries}   
	{\thesubsection}{1em}{} 
\titleformat{\subsubsection}
	{\sffamily\normalsize\bfseries} 
	{\thesubsubsection}{1em}{}

% disponi alberi
\usepackage{forest}

\forestset{
	rectstyle/.style={
		for tree={rectangle,draw,font=\large\sffamily}
	},
	roundstyle/.style={
		for tree={circle,draw,font=\large}
	}
}

% disponi algoritmi
\usepackage{algorithm}
\usepackage{algorithmic}
\makeatletter
\renewcommand{\ALG@name}{Algoritmo}
\makeatother

% disponi numeri di pagina
\usepackage{fancyhdr}
\fancyhf{} 
\fancyfoot[L]{\sffamily{\thepage}}

\makeatletter
\fancyhead[L]{\raisebox{1ex}[0pt][0pt]{\sffamily{\@title \ \@date}}} 
\fancyhead[R]{\raisebox{1ex}[0pt][0pt]{\sffamily{\@author}}}
\makeatother

\begin{document}

% sezione (data)
\section{Lezione del 09-10-24}

% stili pagina
\thispagestyle{empty}
\pagestyle{fancy}

% testo
\subsection{Problema duale ausiliario}
Fino ad ora abbiamo rimandato la trattazione dell'algoritmo per determinare se un poliedro è vuoto.
Diamo adesso quest'algoritmo, notando inoltre che, nel caso il poliedro non fosse vuoto, dovrebbe fornirci un possibile vertice di partenza per l'algoritmo del simplesso.

Partiamo da un poliedro in forma duale standard:
\[
	\begin{cases}
		Ax = b \\
		x \geq 0
	\end{cases}
\]
dove $m$ è il numero di variabili e $n$ il numero di vincoli, e quindi $A: m \times n$.

Adottiamo questa forma in quanto siamo sicuri (a differenza della forma primale) che non presenti linealità.
Costruiamo quindi quello che viene chiamato \textbf{duale ausiliario}, $\text{DA}$:
\[
	\begin{cases}	
		Ax = b \\
		x \geq 0
	\end{cases}
	\rightarrow
	\begin{cases}
		\min \sum_{i=1}^n	\varepsilon_i \\ 
		Ax + I\varepsilon = b \\
		x_i \varepsilon \geq 0
	\end{cases}
\]
dove $\varepsilon$ rappresenta un vettore di variabili ausiliarie di dimensione $n$, cioè una per ogni equazione.
Possiamo rappresentare il sistema ottenuto anche attraverso due matrici a blocchi:
$$
\left( A | I \right) \left( x \over \varepsilon \right) 
$$

Chiamiamo quindi $v(\text{DA})$ la soluzione del duale ausiliario, e formuliamo il teorema:
\begin{theorem}{}
	Se $v(\text{DA}) > 0$, allora $D = \emptyset$, altrimenti, se $v(\text{DA}) = 0$, $D \neq \emptyset$.
	Equivale a dire:
	$$
	v(\text{DA}) = 0 \Leftrightarrow D \neq \emptyset
	$$
\end{theorem}

Dobbiamo però capire come risolvere il duale ausiliario, magari senza usare questo teorema, in quanto questo si andrebbe a creare una catena infinita di duali ausiliari da risolvere...
Notiamo quindi che il duale ausiliario ha sempre una base plausibile, che è quella data dagli indici delle ultime $n$ variabili, quelle introdotte come $\varepsilon_i$.

Non solo, abbiamo anche che:
\begin{theorem}{}
	La soluzione ottima di $\text(DA)$, se $D \neq \emptyset$, ci fornisce un vertice di $D$ stesso.
\end{theorem}
Notiamo che, visto che il primo teorema chiedeva $v(\text{DA}) = 0$ perché $D \neq \emptyset$, allora si ha che nella soluzione ottima di $\text{DA}$ nulla (quella che dimostra la non vuotezza del poliedro), le variabili $\varepsilon_i$ sono nulle.

Il procedimento sarà quindi:
\begin{enumerate}
	\item Prendere il problema duale;
	\item Ricavare il duale ausiliario inserendo nei vincoli il vettore di $n$ variabili ausiliarie $\varepsilon$;
	\item Risolvere il duale ausiliario attraverso il simplesso duale, prendendo come passo iniziale la base data dagli ultimi $n$ indici, cioè che comprende le variabili ausiliarie appena introdotte;
	\item Fare $\geq n$ passi al simplesso, aspettandoci che i primi $n$ passi rimuovano tutte le variabili ausiliarie.
\end{enumerate}

Il vertice che ricaviamo dall'algoritmo prende il nome di \textbf{soluzione ammissibile di base}.

\subsection{Ricavare le variazioni dai cambi di base}
Abbiamo che svolgere un passo al simplesso (primale o duale) significa effettuare un cambio di base, con base entrante e base uscente.
Possiamo calcolare valori definiti, che la soluzione di base sia ammissibile o meno, per la funzione obiettivo del primale e del duale per qualsiasi base.

Potremmo voler calcolare qual'è la variazione di valore di questa funzione dato un certo cambio di base.
Ricordiamo quindi di aver introdotto la formula per spostarci lungo i vincoli di un certo $\lambda$ sul primale:
$$
cx(\lambda) = c\bar{x} - \lambda\bar{y}_h
$$
e sul duale:
$$
by(\lambda) = b\bar{y} + \lambda\left( b_h - A_k \bar{x} \right)
$$

Queste due formule stabiliscono un legame fra il valore della funzione obiettivo e quello di determinate soluzioni $\bar{x}$ e $\bar{y}$ sottoposte a perturbazioni nella direzione dei vertici entranti (cioè allentando il vincolo $h$) di valore $\lambda$.
Possiamo quindi usarle per calcolare quanto di chiedevamo all'inizio del paragrafo: la variazione del valore delle funzioni obiettivo del primale e del duale dopo il cambio di base, ricordando che il nostro $\lambda$ sarà il rapporto $r_i$ (in particolare lo avevamo chiamato $\vartheta$) scelto per determinare il vincolo entrante (primale) o uscente (duale):
$$
r_p = \frac{b_k - A_k \bar{x}}{A_k W^h}, \quad r_d = \frac{-\bar{y}_h}{A_k W^h}
$$

\subsection{Riassunto sulle regole anticiclo}
Abbiamo introdotto, nel corso della trattazione degli algoritmi del simplesso, le \textbf{regole anticiclo di Bland}.
Possiamo riassumerle come segue:
\begin{itemize}
	\item \textbf{\textsf{Simplesso primale}}
		\begin{itemize}
			\item \textbf{Regola anticiclo per l'indice uscente:}
				quando si calcola l'indice uscente $h$ del simplesso primale, si considera una certa soluzione di base complementare duale $\bar{y} = \left( c A_b^-1, 0 \right)$.
				Se consideriamo ogni componente di $\bar{y}$ come il "prezzo ombra" associato ad ogni vincolo, cioè il guadagno (in verità l'opposto del guadagno) che potremmo avere rilassando quel vincolo, sembrerebbe conveniente prendere l'indice del componente più negativo come indice uscente.
				In verità, su certi problemi con soluzioni degeneri questo potrebbe causare cicli infiniti sulle basi degeneri.
				Per questo si adotta la regola anticiclo di prendere il primo indice negativo.
				Così si evitano cicli anche su basi degeneri, cioè ci si assicura che l'algoritmo termina in un numero finito di passi:
		$$
		h := \min\{ i \in B \ \text{t.c.} \ \bar{y_i} < 0 \}
		$$
			\item \textbf{Regola anticiclo per l'indice entrante:} 
				allo stesso modo, quando si calcola l'indice entrante $k$ del simplesso primale, si considerano i rapporti, che rappresentano le distanze sulla base definita dagli indici di base scelti degli altri vincoli, per scegliere il prossimo vincolo più vicino:
				$$
					r_i = \frac{b_i - A_i \bar{x}}{A_iW^h}, \quad \min(r_i) = \vartheta
				$$

				Si potrebbe avere che più indici hanno rapporto che corrisponde a $\vartheta$. 
				Anche in questo caso è necessario scegliere sempre l'indice più piccolo, in quanto questi rappresentano basi degeneri, ovvero un punto che resta vertice su diverse basi, e senza un'apposita regola si potrebbe finire per ciclare:
		$$ 
		k := \min\{ i \in N \ \text{t.c.} \ A_i W^h > 0, \quad \frac{b_i - A_i \bar{x}}{A_i W^h} = \vartheta \} 
		$$
		\end{itemize}
	\item \textbf{\textsf{Simplesso duale}}
		\begin{itemize}
			\item \textbf{Regola anticiclo per l'indice duale:} 
				nel simplesso duale, prima si calcola l'indice entrante $k$ considerando la soluzione di base primale, ovvero prendendo gli $i$ tali che $b_i - A_i \bar{x} < 0$.
				Potrebbero esistere più $i$ che rispettano la proprietà, e bisogna quindi prendere l'indice $i$ più piccolo:
		$$
		k := \min\{ i \in N \ \text{t.c.} \ b_i - A_i \bar{x} < 0 \}
		$$
			\item \textbf{Regola anticiclo per l'indice entrante:} 
				infine, nel simplesso duale, si calcola l'indice uscente $h$ prendendo i rapporti, analoghi a quelli del simplesso primale:
				$$
					r_i = \frac{\bar{y_i}}{A_k W^i}, \quad \min(r_i) = \vartheta
				$$
				anche qui si potrebbero avere più rapporti "distanza" con rapporto $\vartheta$.
				Di questi va nuovamente preso quello con l'indice minore:
		$$ 
		h := \min\{ i \in B \ \text{t.c.} \ A_k W^i < 0, \quad \frac{-\bar{y}_i}{A_kW^i} = \vartheta \}	
		$$
		\end{itemize}
\end{itemize}

In generale, si può dire che in condizioni di sicurezza di non esistenza di basi degeneri, queste regole potrebbero non essere applicare.
In tutti gli altri casi, evitano eventuali cicli sulle stesse basi degeneri.

\subsection{Casi degeneri}
Concludiamo infine la trattazione della PL notando il motivo dell'uso delle regole anticiclo di Bland nel calcolo degli indici uscenti ed entranti nell'algoritmo del simplesso.
Prendiamo un problema di esempio:

\[
	\begin{cases}
		\max(x_1 + x_2) \\ 
		x_1 \leq 1 \\ 
		x_2 \leq 1 \\ 
		x_1 + x_2 \leq 2 \\ 
		x_i \geq 0
	\end{cases}
\]

Si ha che il vertice del poliedro $\bar{x} = (1,1)$ è degenere: possiamo ottenerlo dalle basi $B = \{ 1, 2 \}$, $B = \{ 1, 3 \}$ e $B = \{ 2, 3\}$.
In questo caso tutto funziona perché il vertice è anche ottimo, in caso contrario potremmo, se non si usassero le regole anticiclo di Bland, finire in un ciclo infinito.

Regole alternative a quella di Bland sono la scelta dell'indice sempre maggiore, o degli scarti o rapporti (apparentemente) ottimi, cioè più piccoli:
$$
b_k - A_k \bar{x} = \min_{i \in N}\left( b_i - A_i \bar{x} \right)
$$
Queste regole, sopratutto l'ultima, potrebbero sembrare equivalenti o migliori di quelle di Bland.
Invece è importante ricordare che potrebbero causare cicli.

\end{document}
