
\documentclass[a4paper,11pt]{article}
\usepackage[a4paper, margin=8em]{geometry}

% usa i pacchetti per la scrittura in italiano
\usepackage[french,italian]{babel}
\usepackage[T1]{fontenc}
\usepackage[utf8]{inputenc}
\frenchspacing 

% usa i pacchetti per la formattazione matematica
\usepackage{amsmath, amssymb, amsthm, amsfonts}

% usa altri pacchetti
\usepackage{gensymb}
\usepackage{hyperref}
\usepackage{standalone}

% imposta il titolo
\title{Appunti Ricerca Operativa}
\author{Luca Seggiani}
\date{2024}

% disegni
\usepackage{pgfplots}
\pgfplotsset{width=10cm,compat=1.9}

% imposta lo stile
% usa helvetica
\usepackage[scaled]{helvet}
% usa palatino
\usepackage{palatino}
% usa un font monospazio guardabile
\usepackage{lmodern}

\renewcommand{\rmdefault}{ppl}
\renewcommand{\sfdefault}{phv}
\renewcommand{\ttdefault}{lmtt}

% disponi il titolo
\makeatletter
\renewcommand{\maketitle} {
	\begin{center} 
		\begin{minipage}[t]{.8\textwidth}
			\textsf{\huge\bfseries \@title} 
		\end{minipage}%
		\begin{minipage}[t]{.2\textwidth}
			\raggedleft \vspace{-1.65em}
			\textsf{\small \@author} \vfill
			\textsf{\small \@date}
		\end{minipage}
		\par
	\end{center}

	\thispagestyle{empty}
	\pagestyle{fancy}
}
\makeatother

% disponi teoremi
\usepackage{tcolorbox}
\newtcolorbox[auto counter, number within=section]{theorem}[2][]{%
	colback=blue!10, 
	colframe=blue!40!black, 
	sharp corners=northwest,
	fonttitle=\sffamily\bfseries, 
	title=Teorema~\thetcbcounter: #2, 
	#1
}

% disponi definizioni
\newtcolorbox[auto counter, number within=section]{definition}[2][]{%
	colback=red!10,
	colframe=red!40!black,
	sharp corners=northwest,
	fonttitle=\sffamily\bfseries,
	title=Definizione~\thetcbcounter: #2,
	#1
}

% disponi problemi
\newtcolorbox[auto counter, number within=section]{problem}[2][]{%
	colback=green!10,
	colframe=green!40!black,
	sharp corners=northwest,
	fonttitle=\sffamily\bfseries,
	title=Problema~\thetcbcounter: #2,
	#1
}

% disponi codice
\usepackage{listings}
\usepackage[table]{xcolor}

\lstdefinestyle{codestyle}{
		backgroundcolor=\color{black!5}, 
		commentstyle=\color{codegreen},
		keywordstyle=\bfseries\color{magenta},
		numberstyle=\sffamily\tiny\color{black!60},
		stringstyle=\color{green!50!black},
		basicstyle=\ttfamily\footnotesize,
		breakatwhitespace=false,         
		breaklines=true,                 
		captionpos=b,                    
		keepspaces=true,                 
		numbers=left,                    
		numbersep=5pt,                  
		showspaces=false,                
		showstringspaces=false,
		showtabs=false,                  
		tabsize=2
}

\lstdefinestyle{shellstyle}{
		backgroundcolor=\color{black!5}, 
		basicstyle=\ttfamily\footnotesize\color{black}, 
		commentstyle=\color{black}, 
		keywordstyle=\color{black},
		numberstyle=\color{black!5},
		stringstyle=\color{black}, 
		showspaces=false,
		showstringspaces=false, 
		showtabs=false, 
		tabsize=2, 
		numbers=none, 
		breaklines=true
}

\lstdefinelanguage{javascript}{
	keywords={typeof, new, true, false, catch, function, return, null, catch, switch, var, if, in, while, do, else, case, break},
	keywordstyle=\color{blue}\bfseries,
	ndkeywords={class, export, boolean, throw, implements, import, this},
	ndkeywordstyle=\color{darkgray}\bfseries,
	identifierstyle=\color{black},
	sensitive=false,
	comment=[l]{//},
	morecomment=[s]{/*}{*/},
	commentstyle=\color{purple}\ttfamily,
	stringstyle=\color{red}\ttfamily,
	morestring=[b]',
	morestring=[b]"
}

% disponi sezioni
\usepackage{titlesec}

\titleformat{\section}
	{\sffamily\Large\bfseries} 
	{\thesection}{1em}{} 
\titleformat{\subsection}
	{\sffamily\large\bfseries}   
	{\thesubsection}{1em}{} 
\titleformat{\subsubsection}
	{\sffamily\normalsize\bfseries} 
	{\thesubsubsection}{1em}{}

% disponi alberi
\usepackage{forest}

\forestset{
	rectstyle/.style={
		for tree={rectangle,draw,font=\large\sffamily}
	},
	roundstyle/.style={
		for tree={circle,draw,font=\large}
	}
}

% disponi algoritmi
\usepackage{algorithm}
\usepackage{algorithmic}
\makeatletter
\renewcommand{\ALG@name}{Algoritmo}
\makeatother

% disponi numeri di pagina
\usepackage{fancyhdr}
\fancyhf{} 
\fancyfoot[L]{\sffamily{\thepage}}

\makeatletter
\fancyhead[L]{\raisebox{1ex}[0pt][0pt]{\sffamily{\@title \ \@date}}} 
\fancyhead[R]{\raisebox{1ex}[0pt][0pt]{\sffamily{\@author}}}
\makeatother

\begin{document}

% sezione (data)
\section{Lezione del 10-10-24}

% stili pagina
\thispagestyle{empty}
\pagestyle{fancy}

% testo
\subsection{Introduzione alla programmazione lineare intera}
Iniziamo adesso a studiare un tipo di problemi che finora avevamo menzionato, ma mai risolto.
Si pone il seguente:

\begin{problem}{Caricamento}
	Un ladro è riuscito a scassinare un'importante caveau, e ha di fronte a sé una serie di gemme preziose.
	Tenendo conto del peso e del tipo delle gemme, ricava una tabella col valore e il peso di ogni gemma:
	
	\center \rowcolors{2}{green!10}{green!40!black!20}
	\begin{tabular} { | c || c | c | c | c | c| c |}
		\hline
		Valore & 2 & 4 & 7 & 9 & 13 & 16 \\
		Peso & 6 & 8 & 9 & 11 & 15 & 18 \\
		\hline
	\end{tabular}

	\par\bigskip

	\raggedright
	Amesso che il suo zaino non possa portare più di 30 kg, quali gemme dovrà scegliere per avere il maggiore profitto possibile?
\end{problem}

Questo problema riprende il celebre \textit{knapsack problem}, che risulta essere NP-completo.
Attraverso la \textbf{programmazione lineare intera}, possiamo ricavare soluzioni, almeno approssimate.

Si ha che qualsiasi possibilità delle $2^n$ configurazioni di gemme che il ladro può portare sono rappresentate da un vettore:
$$ 
x = (x_0, ..., x_n), \quad x_i \in \{ 0, 1 \}
$$
e che i il valore e il peso delle stesse sono vettori:
$$
v = \left( 2, 4, 7, 9, 13, 16 \right)
$$
$$
p = \left( 6, 8, 9, 11, 15, 18 \right)
$$

Possiamo quindi formulare il sistema:
\[
	\begin{cases}
		\max \left( 2x_1 + 4x_2 + 7x_3 + 9x_4 + 13x_5 + 16x_6 \right) \\ 
		6x_1 + 8x_2 + 9x_3 + 11x_4 + 15x_5 + 18x_6 \leq 30 \\
		x \in \{ 0, 1 \}^n
	\end{cases}
	\rightarrow
	\begin{cases}
		\max \left( v^T x \right) \\
		p^T x \leq P \\
		x \in \{ 0, 1 \}^n
	\end{cases}
\]
dove $P$ è il vincolo di massimo dei pesi.

Questo problema ricalca la struttura di un problema LP, ma il vincolo finale, che limita i valori possibili delle componenti del vettore soluzione, lo rende IPL.
Decidiamo di trasformare il problema in un LP corrispondente, che chiameremo \textbf{rilassamento continuo}, e risolvere quello.
Possiamo fare ciò in due modi: il primo prevede di rimuovere semplicemente il vincolo, magari includendo la positività, o ancor meglio l'appartenenza $x \in [0, 1]$:
\[
	\begin{cases}
		\max \left( v^T x \right) \\
		p^T x \leq P \\
		0 \leq x \leq 1
	\end{cases}
\]

Questo metodo è sicuramente funzionante, ma preferiamo calcolare il duale, a partire dal primale con imposta la positività:
\[
	\begin{cases}
		\max \left( 2x_1 + 4x_2 + 7x_3 + 9x_4 + 13x_5 + 16x_6 \right) \\ 
		6x_1 + 8x_2 + 9x_3 + 11x_4 + 15x_5 + 18x_6 \leq 30 \\
		-x_1 \leq 0 \\ 
		-x_2 \leq 0 \\ 
		-x_3 \leq 0 \\ 
		-x_4 \leq 0 \\ 
		-x_5 \leq 0 \\ 
		-x_6 \leq 0
	\end{cases}
	\rightarrow 
	\begin{cases}
		\min \left( 30y_1 \right) \\ 
		6 y_1 - y_2 = 2 \\
		8 y_1 - y_3 = 4 \\ 
		9 y_1 - y_4 = 7 \\ 
		11 y_1 - y_5 = 9 \\ 
		15 y_1 - y_6 = 13 \\ 
		18 y_1 - y_7 = 16 \\ 
		y_i \geq 0
	\end{cases}
\]
da cui notiamo che le disequazioni in forma $a_i x + s_i = b_i$ si possono rendere come $a_i x \geq b_i$: 
\[
	\begin{cases}
		\min \left( 30y_1 \right) \\ 
		6 y_1 \geq 2 \\
		8 y_1 \geq 4 \\ 
		9 y_1 \geq 7 \\ 
		11 y_1 \geq 9 \\ 
		15 y_1 \geq 13 \\ 
		18 y_1 \geq 16 \\ 
		y_1 \geq 0
	\end{cases} \equiv
	\begin{cases}
		\min \left( 30y_1 \right) \\ 
		y_1 \geq \frac{1}{3} \\
		y_1 \geq \frac{1}{2} \\ 
		y_1 \geq \frac{7}{9} \\ 
		y_1 \geq \frac{9}{11} \\ 
		y_1 \geq \frac{13}{15} \\ 
		y_1 \geq \frac{8}{9} \\ 
		y_1 \geq 0
	\end{cases}
\]

Abbiamo che questo sistema è banale: il minimo di $30 y_1$ sottoposto ai vincoli si avrà quando $y_1$ rispetta i vincoli, che sono tutti $\geq$, e rispettare qualsiasi vincolo significa rispettare il vincolo con $b_i$ più grande, quindi in questo caso $\frac{8}{9}$
Si ha quindi che la soluzione ottima del duale è $\frac{8}{9}$, e possiamo trovare con la stessa facilità la soluzione del primale: abbiamo che la disequazione da cui si è ricavato la soluzione del duale ha indice $j = 6$.
Si prende quindi la variabile con lo stesso indice nel primale, e si \textbf{satura}, cioè si porta al livello più alto possibile prima di violare il limite.
Si nota che il valore ottimo trovato nel primale e nel duale (che è uguale dalla dualità forte) è $v = P \bar{y}_j = v_j \bar{x}_j $, con $\bar{x}$ e $\bar{y}$ le soluzioni ottime trovate rispettivamente al primale e al duale. 

Nel nostro caso, $18 x_6 \leq 30$, quindi $x_6 = \frac{30}{18}$, con valore $v = 16 \frac{30}{18} = \frac{80}{3}$.

Questo approccio sarebbe quello che si applicherebbe intuitivamente senza sapere nulla di PL. 
Infatti i rapporti calcolati nel duale:
$$
r_i = \frac{v_i}{p_i}
$$
sono effettivamente i \textbf{rendimenti} di ogni elemento di peso e valore, cioè quanto valore portano in rapporto al loro peso.
Chiaramente, vorremmo massimizzare gli elementi con rendimento maggiore, quindi saturiamo gli elementi con $r_i$ massimo.
Possiamo quindi definire il seguente algoritmo:

\begin{algorithm}[H]
\caption{caricamento per soluzioni ottime del rilassato continuo}
\begin{algorithmic}
	\STATE \textbf{Input:} il rilassato continuo di un problema di caricamento
	\STATE \textbf{Output:} la soluzione ottima $\bar{x}$
	\STATE Considera i $r_i = \frac{v_i}{p_i}$ rendimenti di ogni elemento
	\STATE Scegli l'indice $j$ del rendimento massimo $r_j$
	\STATE Satura la variabile $j$
\end{algorithmic}
\end{algorithm}

Abbiamo che la soluzione trovata è un limite superiore per il problema di partenza, quello ILP: infatti, rilassando i vincoli, otteniamo una regione ammissibile più grande, e quindi aumentiamo il massimo.
In verità anche l'arrotondamento in basso della soluzione $v$ è un limite superiore: nel caso del problema abbiamo $ \lfloor \frac{80}{3} \rfloor = 26$.

Un'idea banale per trovare il punto ottimo di questo problema ILP potrebbe essere quello di arrotondare il punto ottimo del rilassato continuo: nel nostro esempio, troviamo $\bar{x} = \left( 0, 0, 0, 0, 0, 1 \right)$, che dà massimo $v = 16$.

Notiamo un risultato fondamentale: se chiamiamo il valore arrotondato della soluzione del rilassato continuo $v_s$, e il valore ottenuto nell'approssimazione del \textit{punto} di ottimo $v_i$, è vero che:
$$ v_i \leq v_{ILP} \leq v_p $$
dove $v_{ILP}$ è la soluzione del problema ILP che stiamo cercando.
Per la precisione, possiamo definire l'errore:
$$
\varepsilon = \frac{v_s - v_i}{v_i}
$$
che sul problema in esame dà $\varepsilon \approx 60\%$, chiaramente poco utile. 
Sarà necessario adottare un qualche metodo per migliorare la stima data dall'arrotondamento del punto di ottimo del rilassato continuo.

Per adesso, generalizziamo quando trovato finora: bisogna distinguere fra $v_i$ e $v_s$ che si parli di problemi di minimizzazione o di massimizzazione.
Si ha infatti che:
\begin{theorem}{Bound di problemi ILP}
	Per un dato problema ILP con soluzione $v_{ILP}$, si ha che:
	$$
		v_i \leq v_{PLI} \leq v_s
	$$
	\begin{itemize}
		\item Se il problema è di \textbf{massimizzazione}, $v_i$ è il valore calcolato soluzione ammissibile del rilassato continuo arrotondata \textbf{per difetto}, e $v_s$ è la soluzione ottima del rilassato continuo arrotondata \textbf{per difetto};
		\item Se il problema è di \textbf{minimizzazione}, $v_i$ è la soluzione ottima del rilassato continuo arrotondata \textbf{per eccesso}, e $v_i$ è il valore calcolato soluzione ammissibile del rilassato continuo arrotondata \textbf{per eccesso};
	\end{itemize}
\end{theorem}

\subsubsection{Vincoli booleani}
Possiamo chiamare il vincolo introdotto prima, $x \in \{ 0, 1 \}^n$, \textbf{vincolo booleano} (in modo più o meno informale).
Come abbiamo detto, questo vincolo può essere reso come $x_i \geq 0$, o ancor meglio $0 \geq x_i \geq 1$ quando si porta al rilassato continuo (abbiamo usato il primo insieme di vincoli per ricavare il duale del rilassato continuo).

Introduciamo quindi l'algoritmo, equivalente a quello presentato prima per :
\begin{algorithm}[H]
\caption{caricamento per soluzioni ottime del rilassato continuo con vincolo booleano}
\begin{algorithmic}
	\STATE \textbf{Input:} il rilassato continuo di un problema con vincolo booleano
	\STATE \textbf{Output:} la soluzione ottima $\bar{x}$
	\STATE Considera i $r_i = \frac{v_i}{p_i}$ rendimenti di ogni elemento
	\STATE Scegli l'indice $j$ del rendimento massimo $r_j$
	\STATE Satura la variabile $j$
	\STATE Quando finisci lo spazio, satura con il bene rimanente (sic.)
\end{algorithmic}
\end{algorithm}

\end{document}
