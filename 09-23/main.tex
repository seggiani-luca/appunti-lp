\documentclass[a4paper,11pt]{article}
\usepackage[a4paper, margin=8em]{geometry}

% usa i pacchetti per la scrittura in italiano
\usepackage[french,italian]{babel}
\usepackage[T1]{fontenc}
\usepackage[utf8]{inputenc}
\frenchspacing 

% usa i pacchetti per la formattazione matematica
\usepackage{amsmath, amssymb, amsthm, amsfonts}

% usa altri pacchetti
\usepackage{gensymb}
\usepackage{hyperref}
\usepackage{standalone}

% imposta il titolo
\title{Appunti Ricerca Operativa}
\author{Luca Seggiani}
\date{23-09-24}

% imposta lo stile
% usa helvetica
\usepackage[scaled]{helvet}
% usa palatino
\usepackage{palatino}
% usa un font monospazio guardabile
\usepackage{lmodern}

\renewcommand{\rmdefault}{ppl}
\renewcommand{\sfdefault}{phv}
\renewcommand{\ttdefault}{lmtt}

% disponi teoremi
\usepackage{tcolorbox}
\newtcolorbox[auto counter, number within=section]{theorem}[2][]{%
	colback=blue!10, 
	colframe=blue!40!black, 
	sharp corners=northwest,
	fonttitle=\sffamily\bfseries, 
	title=Teorema~\thetcbcounter: #2, 
	#1
}

% disponi definizioni
\newtcolorbox[auto counter, number within=section]{definition}[2][]{%
	colback=red!10,
	colframe=red!40!black,
	sharp corners=northwest,
	fonttitle=\sffamily\bfseries,
	title=Definizione~\thetcbcounter: #2,
	#1
}

% disponi problemi
\newtcolorbox[auto counter, number within=section]{problem}[2][]{%
	colback=green!10,
	colframe=green!40!black,
	sharp corners=northwest,
	fonttitle=\sffamily\bfseries,
	title=Problema~\thetcbcounter: #2,
	#1
}

% disponi codice
\usepackage{listings}
\usepackage[table]{xcolor}

\lstdefinestyle{codestyle}{
		backgroundcolor=\color{black!5}, 
		commentstyle=\color{codegreen},
		keywordstyle=\bfseries\color{magenta},
		numberstyle=\sffamily\tiny\color{black!60},
		stringstyle=\color{green!50!black},
		basicstyle=\ttfamily\footnotesize,
		breakatwhitespace=false,         
		breaklines=true,                 
		captionpos=b,                    
		keepspaces=true,                 
		numbers=left,                    
		numbersep=5pt,                  
		showspaces=false,                
		showstringspaces=false,
		showtabs=false,                  
		tabsize=2
}

\lstdefinestyle{shellstyle}{
		backgroundcolor=\color{black!5}, 
		basicstyle=\ttfamily\footnotesize\color{black}, 
		commentstyle=\color{black}, 
		keywordstyle=\color{black},
		numberstyle=\color{black!5},
		stringstyle=\color{black}, 
		showspaces=false,
		showstringspaces=false, 
		showtabs=false, 
		tabsize=2, 
		numbers=none, 
		breaklines=true
}

\lstdefinelanguage{javascript}{
	keywords={typeof, new, true, false, catch, function, return, null, catch, switch, var, if, in, while, do, else, case, break},
	keywordstyle=\color{blue}\bfseries,
	ndkeywords={class, export, boolean, throw, implements, import, this},
	ndkeywordstyle=\color{darkgray}\bfseries,
	identifierstyle=\color{black},
	sensitive=false,
	comment=[l]{//},
	morecomment=[s]{/*}{*/},
	commentstyle=\color{purple}\ttfamily,
	stringstyle=\color{red}\ttfamily,
	morestring=[b]',
	morestring=[b]"
}

% disponi sezioni
\usepackage{titlesec}

\titleformat{\section}
	{\sffamily\Large\bfseries} 
	{\thesection}{1em}{} 
\titleformat{\subsection}
	{\sffamily\large\bfseries}   
	{\thesubsection}{1em}{} 
\titleformat{\subsubsection}
	{\sffamily\normalsize\bfseries} 
	{\thesubsubsection}{1em}{}

% disponi alberi
\usepackage{forest}

\forestset{
	rectstyle/.style={
		for tree={rectangle,draw,font=\large\sffamily}
	},
	roundstyle/.style={
		for tree={circle,draw,font=\large}
	}
}

% disponi algoritmi
\usepackage{algorithm}
\usepackage{algorithmic}
\makeatletter
\renewcommand{\ALG@name}{Algoritmo}
\makeatother

% disponi numeri di pagina
\usepackage{fancyhdr}
\fancyhf{} 
\fancyfoot[L]{\sffamily{\thepage}}

\makeatletter
\fancyhead[L]{\raisebox{1ex}[0pt][0pt]{\sffamily{\@title \ \@date}}} 
\fancyhead[R]{\raisebox{1ex}[0pt][0pt]{\sffamily{\@author}}}
\makeatother

% disegni
\usepackage{pgfplots}
\pgfplotsset{width=10cm,compat=1.9}

\begin{document}
% sezione (data)
\section{Lezione del 23-09-24}

% stili pagina
\thispagestyle{empty}
\pagestyle{fancy}

% testo
\subsection{Introduzione}

\subsubsection{Programma del corso}
Il corso di ricerca operativa si divide in 4 parti:

\begin{enumerate}
	\item Modello di Programmazione Lineare;
	\item Programmazione Lineare su reti, ergo programmazione lineare su grafi;
	\item Programmazione Lineare intera, ergo programmazione lineare col vincolo $x \in \mathbb{Z}^n$;
	\item Programmazione Non Lineare.
\end{enumerate}

Le prime 3 parti hanno come prerequisiti l'algebra lineare: in particolare operazioni matriciali, prodotti scalari, sistemi lineari, teorema di Rouché-Capelli.
La quarta parte richiede invece conoscenze di Analisi II.

\subsubsection{Un problema di programmazione lineare}

La ricerca operativa si occupa di risolvere problemi di ottimizzazione con variabili decisionali e risorse limitate.
Poniamo un problema di esempio:

\begin{problem}{Produzione}
Una ditta produce due prodotti: \textbf{laminato A} e \textbf{laminato B}.
Ogni prodotto deve passare attraverso diversi reparti: il reparto \textbf{materie prime}, il reparto \textbf{taglio}, il reparto \textbf{finiture A} e il reparto \textbf{finiture B}.
Il guadagno è rispettivamente di 8.4 e 11.2 (unità di misura irrilevante) per ogni tipo di laminato.

Ora, nel reparto materie prime, il laminato A occupa 30, ore, e lo B 20 ore.
Nel reparto taglio il laminato A occupa 10 ore e lo B 20 ore.
Il laminato A occupa poi 20 ore nel reparto finiture A, mentre il laminato B occupa 30 ore nel reparto finiture B.
I reparti hanno a disposizione, rispettivamente, 120, 80, 62 e 105 ore.
Possiamo porre queste informazioni in forma tabulare:

	\center \rowcolors{2}{green!10}{green!40!black!20}
	\begin{tabular} { | c || c | c | c | }
		\hline
		\bfseries Reparto & \bfseries Capienza & \bfseries Laminato A & \bfseries Laminato B \\
		\hline 
		Materie prime & 120 & 30 & 20 \\
		Taglio & 80 & 10 & 20 \\
		Finiture A & 62 & 20 & / \\
		Finiture B & 105 & / & 30 \\
		\hline
		\textbf{Guadagno} & & 8.4 & 11.2 \\
		\hline
	\end{tabular}

	\par\bigskip

Quello che ci interessa è chiaramente massimizzare il guadagno.
\end{problem}

Decidiamo di modellizzare questa situazione con un modello matematico.

Il guadagno che abbiamo dai laminati rappresenta una \textbf{funzione obiettivo}, ovvero la funzione che vogliamo ottimizzare.
Ottimizzare significa trovare il modo migliore di massimizzare o minimizzare i valori della funzione agendo sulle variabili decisionali.
La funzione obiettivo va ottimizzata rispettando determinati \textbf{vincoli}, che modellizzano il fatto che le risorse sono limitate.
Una \textbf{soluzione ammissibile} è una qualsiasi soluzione che rispetta i vincoli del problema.
Chiamiamo quindi \textbf{regione ammissibile} l'insieme di tutte le soluzioni ammissibili.
All'interno della regione ammissibile c'è la soluzione che cerchiamo, ovvero la \textbf{soluzione ottima}.

Decidiamo quindi le \textbf{variabili decisionali}, ed esplicitiamo la funzione obiettivo e i vincoli.

In questo caso le variabili decisionali saranno le quantità di laminato A e B da produrre, che individuano un punto in $ \mathbb{R}^2 $ denominato $ ( x_A, x_B ) $. 
Decidere di usare la soluzione $ (1,1) $ significa decidere di produrre 1 unità di laminato A e 1 unità di laminato B, per un guadagno complessivo di $ 8.4 + 11.2 = 19.6 $.

La funzione obiettivo sarà quindi:

$$ f(x_A, x_B) = 8.4 x_A + 11.2 x_B, \quad f: \mathbb{R}^2 \rightarrow \mathbb{R} $$

lineare, e noi saremo interessati a:

$$ \max(f(x_A, x_B)) $$

rispettando i vincoli, ergo nella regione ammissibile.
Per esprimere questi vincoli, cioè il tempo limitato all'interno di ogni reparto, introduciamo il sistema di disequazioni:

\[
	\begin{cases}
		30 x_A + 20 x_B \leq 120 \\
		10 x_A + 20 x_B \leq 80	\\
		20 x_A + 0 x_B \leq 62 \\	
		0 x_A + 30 x_B \leq 105 \\
		- x_A \leq 0 \\
		- x_B \leq 0 \\
	\end{cases}
\]

dove notiamo le ultime due disequazioni indicano la positività di $x_A$ e $x_B$, in forma $ f(x_A, x_B) \leq b $.
Questo sistema non indica altro che la regione ammissibile.

Possiamo riscrivere questo modello usando la notazione dell'algebra lineare.
La funzione obiettiva e i vincoli diventano semplicemente:

\[
	\begin{cases}
		\max(c^T \cdot x) \\
		A \cdot x \leq b	
	\end{cases}
\]

dove $c$ rappresenta il vettore dei costi, $A$ rappresenta la matrice dei costi a $b$ il vettore dei vincoli.
$c$ è trasposto per indicare prodotto fra vettori.

Possiamo scrivere $A$, $b$ e $c$ per esteso:

$$
A:
\begin{pmatrix}
	30 & 20 \\
	10 & 20 \\
	20 & 0 \\
	0 & 30 \\
	-1 & 0 \\
	0 & -1
\end{pmatrix}, \quad
b:
\begin{pmatrix}
	120 \\
	80 \\
	62 \\ 
	105 \\ 
	0 \\ 
	0 
\end{pmatrix}, \quad 
c:
\begin{pmatrix}
	8.4 \\
	11.2 \\
\end{pmatrix}
$$

Notiamo come $A$ e $b$ hanno dimensione verticale $ 4 + 2 = 6 $, dai 4 vincoli superiori e i 2 vincoli inferiori.

A questo punto, possiamo disegnare la regione ammissibile come l'intersezione dei semipiani individuati da ogni singola disuguaglianza.

Si riporta un grafico:

\begin{center}

\begin{tikzpicture}
\begin{axis}[
    axis lines = middle,
    xlabel = {$x_A$},
    ylabel = {$x_B$},
    xmin=0, xmax=7.9,
    ymin=0, ymax=3.9,
    domain=0:10,
    samples=100,
    width=15cm, height=7cm,
    legend pos=north east
  ]

% regione ammissibile

	\addplot[fill=gray, opacity=0.4, forget plot] 
    coordinates {
			(0, 0)
			(3.1, 0)
			(3.1, 1.35)
			(2,3)
			(1, 3.5)
			(0, 3.5)
		};

% rette

\addplot[domain=2:3.1, thick, blue] {6 - 1.5*x}; 
\addlegendentry{$ 30 x_A + 20 x_B \leq 120 $}

\addplot[domain=1:2, thick, green] {4 - 0.5*x}; 
\addlegendentry{$ 10 x_A + 20 x_B \leq 80 $}

\addplot[thick, purple] coordinates {(3.1, 0) (3.1, 1.35)};
\addlegendentry{$ 20 x_A + 0 x_B \leq 62 $}

\addplot[domain=0:1, thick, red] {3.5}; 
\addlegendentry{$ 0 x_A + 30 x_B \leq 105 $}
	
\end{axis}
\end{tikzpicture}

\end{center}

In diversi colori sono riportati i margini delle disequazioni, mentre in grigio è evidenziata la regione ammissibile.
Qualsiasi punto all'interno della regione ammissibile vale come soluzione, e almeno uno di essi è soluzione ottimale.

\par\smallskip

Il modello finora descritto prende il nome di modello di programmazione lineare, e permette di formulare problemi di programmazione lineare (LP).

\begin{definition}{Problema di programmazione lineare (1)}
Un problema di programmazione lineare (LP) riguarda l'ottimizzazione di una funzione lineare in più variabili
soggetta a vincoli di tipo $ =, \ \leq $ e $ \geq $, ovvero in forma:
\[
	\begin{cases}
			\min / \max(c^T \cdot x) \\
			A_i x \leq b \\
			B_j x \geq d \\
			C_k x = e \\
	\end{cases}
\]
\end{definition}

"Programmazione" qui non ha alcun legame col concetto di programmazione informatica, ma si riferisce al fatto che il modello è effettivamente \textit{programmabile}.

"Lineare" si riferisce alla linearità dei problemi che ci permette di risolvere (e quindi del modello).




\end{document}
